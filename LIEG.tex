\PassOptionsToPackage{usenames,svgnames,dvipsnames,table}{xcolor}
\documentclass[11pt,a4paper]{scrarticle}

\usepackage[english]{babel}
% Named colors
\usepackage[dvipsnames, table, xcdraw]{xcolor}

% Math stuff
\usepackage{amsmath, amsthm,  amsfonts, amssymb}
\usepackage{tikz, tikz-cd, graphicx}
\usepackage{quiver} 
\usepackage{mathrsfs, mathtools, bm}
\usepackage{halloweenmath}
\usepackage{fullpage}
% Refer
\usepackage[colorlinks,citecolor=blue,urlcolor=blue,bookmarks=false,hypertexnames=true]{hyperref} 
\usepackage[capitalize]{cleveref}

% Change numbering in enumerate
\usepackage[shortlabels]{enumitem}
\crefname{enumi}{part}{parts}
\crefname{figure}{Figure}{Figures}


% pagesetup
\usepackage{graphicx}
%\usepackage{fancyhdr}
%\pagestyle{fancy}
%\setlength{\headheight}{0.75in}
%\setlength{\oddsidemargin}{0in}
%\setlength{\evensidemargin}{0in}
%\setlength{\voffset}{-1.0in}
%\setlength{\headsep}{10pt}
%\setlength{\textwidth}{6.5in}
%\setlength{\headwidth}{6.5in}
%\setlength{\textheight}{8.75in}
%\setlength{\parskip}{1ex plus 0.5ex minus 0.2ex}
%\setlength{\footskip}{0.3in}
%\fancyhead[L]{\textbf{Devesh Rajpal}}
%\fancyhead[R]{\nouppercase\leftmark}
\usepackage{multicol, titletoc, bookmark}


% Colored boxes
\usepackage{thmtools}
\usepackage[framemethod=TikZ]{mdframed}
\mdfsetup{skipabove=1em,skipbelow=0em}

\theoremstyle{definition}
\declaretheoremstyle[
    headfont=\bfseries\sffamily\color{ForestGreen!70!black}, bodyfont=\normalfont,
    mdframed={
        linewidth=2pt,
        rightline=false, topline=false, bottomline=false,
        linecolor=ForestGreen, backgroundcolor=ForestGreen!5,
    }
]{greenbox}

\declaretheoremstyle[
    headfont=\bfseries\sffamily\color{NavyBlue!70!black}, bodyfont=\normalfont,
    mdframed={
        linewidth=2pt,
        rightline=false, topline=false, bottomline=false,
        linecolor=NavyBlue, backgroundcolor=NavyBlue!5,
    }
]{bluebox}

\declaretheoremstyle[
    headfont=\bfseries\sffamily\color{Mulberry!70!black}, bodyfont=\normalfont,
    mdframed={
        linewidth=2pt,
        rightline=false, topline=false, bottomline=false,
        linecolor=Mulberry, backgroundcolor=Mulberry!5,
    }
]{purplebox}

\declaretheoremstyle[
    headfont=\bfseries\sffamily\color{RawSienna!70!black}, bodyfont=\normalfont,
    mdframed={
        linewidth=2pt,
        rightline=false, topline=false, bottomline=false,
        linecolor=RawSienna, backgroundcolor=RawSienna!5,
    }
]{redbox}

\declaretheoremstyle[
    headfont=\bfseries\sffamily\color{CadetBlue!70!black}, bodyfont=\normalfont,
    numbered=no,
    mdframed={
        linewidth=2pt,
        rightline=false, topline=false, bottomline=false,
        linecolor=CadetBlue, backgroundcolor=CadetBlue!4,
    },
    qed=\qedsymbol
]{proofbox}
\declaretheoremstyle[
    headfont=\bfseries\sffamily\color{CadetBlue!70!black}, bodyfont=\normalfont,
    mdframed={
        linewidth=2pt,
        rightline=false, topline=false, bottomline=false,
        linecolor=CadetBlue, backgroundcolor=CadetBlue!10,
    }
]{CadetBluebox}


\renewenvironment{proof}[1][\proofname]{\vspace{-10pt}\begin{myproof}}{\end{myproof}}
\declaretheorem[style = CadetBluebox, name = Theorem, numberwithin=section]{thm}
%\declaretheorem[style = proofbox, name=Proof, numbered=no]{myproof}
\declaretheorem[name=Proof, numbered=no,qed=\qedsymbol]{myproof}
\declaretheorem[style = greenbox, name = Definition, numberwithin=section]{defn}
\declaretheorem[style = greenbox, name = Notation]{notation}
\declaretheorem[style = redbox, name = Proposition,sibling=thm]{proposition}

\declaretheorem[style = CadetBluebox, name = Lemma, sibling=thm]{lemma}
\declaretheorem[style = redbox, name = Corollary, numbered = no]{corollary}
\declaretheorem[style = bluebox, name = Observation, numbered = no]{observation}
\declaretheorem[style = bluebox, name = Example, numbered = no]{example}
\declaretheorem[style = bluebox, name = Remark, numbered = no]{remark}
\declaretheorem[style = purplebox, name = Problem, numbered = no]{problem}
\declaretheorem[style = purplebox, name = Exercise, numbered = no]{exercise}
\theoremstyle{greenbox}
\newtheorem*{motivation}{Motivation}
\newtheorem*{recall}{Recall}
\newtheorem*{note}{Note}


% Figure support.
\usepackage{pdfpages}
\usepackage{transparent}
\graphicspath{{figures/}{images/}} %typical directories

%math
\newcommand{\R}{\mathbb{R}}
\newcommand{\Z}{\mathbb{Z}}
\newcommand{\Q}{\mathbb{Q}}
\newcommand{\C}{\mathbb{C}}
\newcommand{\Ho}{\mathbb{H}}
\newcommand{\T}{\mathbb{T}}
\newcommand{\dou}{\partial}
\newcommand{\half}{\ensuremath{\frac{1}{2}}}
\newcommand{\pt}{\partial_t}
\newcommand{\dt}{\frac{\partial}{\partial_t}}
\newcommand{\define}{\doteqdot}
\newcommand{\Ch}{\text{Ch}}

\newcommand{\ag}{\mathfrak{a}}
\newcommand{\fg}{\mathfrak{g}}
\newcommand{\hg}{\mathfrak{h}}
\newcommand{\zg}{\mathfrak{z}}

\title{\LARGE Lie Groups}
\author{\large Devesh Rajpal}
\date{}
\pdfsuppresswarningpagegroup=1
\begin{document}
    \maketitle
    \tableofcontents
    \section{4th January 23}
    One can study Lie Groups from several points of view. The course is aimed to understand the structure of Lie Groups. 
    \begin{defn}
        A smooth manifold $ M $ is a Hausdorff space which is locally Euclidean with a smooth atlas i.e. (i) given any $ x \in M $, $ \exists  $ a chart $ (U, \phi) , \, x \in U \subset M$ with  $ \phi : U \to \phi (U) $ open in $ \R^{m} $.


        (ii) We have collection $ \{(U, \phi)\} $ of charts such that 
        \[ \phi ( U \cap V) \xrightarrow{\psi \circ \phi^{-1}} \psi ( U \cap V)\]
        is  a diffeomorphism.
        
    \end{defn}

    Suppose $ f : M \to N $ is a continuous map between manifolds. We say that $  f  $ is smooth if for $ (U, \phi) \in \mathcal{q}(M) $, $ (V, \psi) \in \mathcal{q}(N) $ such that $ f(U) \subset V $ and $ \psi \circ f \circ  \phi^{-1} $ is smooth. 
    
    TO DO : Construction of tangent bundle and vector bundle

    \section{9th Jan 2023}

    \begin{defn}
        $ G $ is a Lie group if \begin{enumerate}
            \item $ G $ is a smooth manifold
            \item $ G $ is also a group s.t 
            \begin{align*}
                \mu : G \times G \to G \\
                ( g,h) \mapsto gh
            \end{align*} and 
            \begin{align*}
                i : G \to G \\
                g \mapsto g^{-1}
            \end{align*}
            are smooth maps.
        \end{enumerate}
    \end{defn}
    \begin{defn}
        A real (or complex) vector space $ V $ together with a bilinear map \begin{align*}
            [,] : V \times V \to V
        \end{align*}
        is called a \textbf{Lie Algebra} if \begin{enumerate}
            \item $[X,Y] = -[Y,X]$ - skew symmetry
            \item $[[X,Y],Z]+ [[Y,Z],X]+ [[Z,X],Y] = 0 $ - Jacobi identity
        \end{enumerate}
    \end{defn}

    \begin{example}
        \begin{enumerate}
            \item $ \left( \R, + \right) , (\C, +)$, $ V $ any f.d vector space over $ \R $ or $ \C $. 
            \item $ (\R^{\times}, \cdot), (\C^{\times}, \cdot) $
            \item $ S^{1} = \{ z \in \C^{\times}| |z| =1\} $
            \item $ \operatorname{GL}_{n}(\R) $, $ \operatorname{GL}_{n}(\C) $
            \item $ \R^{n} / \Z^{n} \cong \left( \R^{n} /\Z^{n} \right) \cong (S^{1})^{n}$
            \item Suppose $ \Gamma \subset V $ is a discrete subgroup. Then $ V / \Gamma  $ is a Lie group.
            \item $ N = $ unipotent upper triangular matrices, $ B =  $ upper triangular matrices. As manifolds $ N \cong \R^{\binom{n}{2}} $ and $ B \cong (\R^{\times})^{n} \times N $.
            \item $ \operatorname{SL}_{n}(\R) = \{X \in \operatorname{GL}_{n}(\R) | \det X = 1\} $, $ \operatorname{SL}_{n}(\C) $. 
            \item $ O(n) $, $ SO(n) $.
            \item $ U(n) $, $ SU(n) $.
            \item $ \Ho^{\times} $, $ S^{3}  $ with quaternion multiplication.
            \item $ Sp(n) = \{X \in \operatorname{GL}_{n}(\R)| X \text{ preserves quaternion structure as a subset of } \operatorname{Aut}_{\Ho}\Ho^{n}\} $
        \end{enumerate}
    \end{example}
    \begin{problem}
        $ V / \Gamma  \cong \R^{k} \times  (S^{1})^{n-k}$ for $ n $-dimensional vector space $  V$. 
    \end{problem}
    \begin{thm}
        Suppose $ G $ is a compact, connected, simple Lie group. Then $ G $ is locally isomorphic to \begin{enumerate}
            \item $ SU(n) , n \ge 2$ denoted by $ A_{n-1} $
            \item $ SO(2n+1), n \ge 2 $ denoted by $ B_{n} $
            \item $ Sp(n), n \ge 1 $ denoted by $ C_{n} $
            \item $ SO(2n), n\ge 2 $ denoted by $ D_{n} $
        \end{enumerate}
        or one of the following exceptional Lie group $ G_{2}, F_{4}, E_{6}, E_{7}, E_{8} $.

    \end{thm}
    \begin{problem}
        Prove that $ \operatorname{SL}_{n}(\R) $ and $ O(n) $ are smooth manifold, hence Lie groups.
    \end{problem}

    Examples of Lie algebra - \begin{example}
       \begin{enumerate}
        \item $ (V, [\cdot, \cdot] \equiv 0 )$ is called trivial Lie algebra. 
        \item $ (\mathfrak{gl}_{n}(\R), [A,B ]  = AB - BA )$, $ \mathfrak{gl}_{n}(\C)$
        \item $ \mathfrak{sl}_n(\R) $ ($ \mathfrak{sl}_n (\C) $) is the Lie subalgebra of $ \mathfrak{gl}_{n}(\R) $ ($ \mathfrak{gl}_{n}(\C) $)  consisting of trace $ 0 $.
        \item $ \mathfrak{so}_n $ is Lie subalgebra of $ \mathfrak{gl}_{n}(\R) $ consisting of skew-symmetric matrices.
       \end{enumerate}
    \end{example}
    \begin{defn}
        A vector field $ X $ on a Lie group $ G $ is called left invariant if $ (L_{g})_{*}(X_{h}) = X_{gh} $
    \end{defn}

    \section{11th Jan 2023}

    Recall $ \Ho =  \{a+bi+cj+dk : (a,b,c,d) \in \R^{4},  \, i^{2} = -1, j^{2} = -1, k^{2} = -1,  ij=k, jk = l, ki=j \} $ is the quaternion division algebra  with the norm 
    \[ ||a+ bi+ cj+dk||^{2} = a^{2}+ b^{2}+c^{2}+d^{2} \] which satisfies $ ||q_{1} \cdot  q_{2}|| = ||q_{1}|| \cdot ||q_{2}||$

    We can put this multiplication on $ S^{3} \cong SU(2) $ to get a compact Lie group. To get the isomorphism $ SU(2) \cong S^{3} $, we define a map 
    \begin{align*}
        \phi :S^{3} & \to SU(2)\\
            (a,b,c,d) & \mapsto \begin{bmatrix}
                a+bi & c+di\\
                -(c-di)&  a-bi
            \end{bmatrix}
    \end{align*}
    which is an algebra isomorphism. 

    \begin{defn}
        The Lie algebra of $ G $ is the space of all left-invariant vector fields on $ G $. 
        
    \end{defn}
    We have an  isomorphism \begin{align*}
        \mathfrak{g}  = \operatorname{Lie}(G) &\to T_{e}G\\ 
        X &\mapsto X_{e}
    \end{align*}
    \begin{example}
        Let $ G = \R^{n} $, with identity element $ 0 \in \R^{n} $ and left-invariant vector fields $ \{ \frac{ \partial}{ \partial x_{1}}, \ldots, \frac{ \partial}{ \partial x_{n}}\} $. Then the Lie bracket is \begin{align*}
            [\cdot, \cdot ] \equiv 0
        \end{align*}
    \end{example}
    \begin{remark}
        In general for any abelian Lie group $ G $, the Lie bracket is $ [\cdot, \cdot ] \equiv 0 $.
    \end{remark}
    \begin{thm}
        Let $ G $ be a connected Lie group. Then \begin{enumerate}
            \item Lie$ (G)  = \mathfrak{g}$ is isomorphic as a vector space to $ T_{e}(G) $.
            \item Left-invariant vector fields are smooth. 
            \item Lie$ (G) $ is closed under Lie bracket. 
        \end{enumerate}
    \end{thm}
    \begin{proof} 1. 
        Let $ X $ be a left-invariant vector field on $ G $. We need to show that $ Xf $ is smooth for each $ f \in C^{\infty}(G) $. 
        \begin{align*}
            (Xf)(g) & = X_{g}f\\
             & = (d \lambda_{g} X_{e})f \\
             & = X_{e}( f \circ \lambda_{g})
        \end{align*}
        To show that $ Xf $ is smooth, it suffices to show that $ X_{e}(f \circ \lambda_{g}) $ is smooth. We realize $ X_{e}(f \circ \lambda_{g}) $ as evaluation of a smooth function on a smooth function. 

        Let $ Y $ be a smooth vector field on $ G $ such that $ Y_{e} = X_{e}$ \begin{align*}
            Y_{e}(f \circ \lambda_{g}) = X_{e}( f \circ \lambda_{g})
        \end{align*}
        We look at $ \lambda_{g} $ as the composition of \begin{align*}
            G &\xrightarrow{i_{g}^{2}} G \times G 
            \xrightarrow{\mu} G \\
            x & \mapsto (g,x)\mapsto gx
        \end{align*}
        Regard $ Y $ as the vector field $ (0,Y) $ on $ G \times G $. Now \begin{align*}
            (0,Y)(f \circ \mu) \circ i_{e}^{1}(g) & = (0,Y)_{(g,e)} ( f\circ \mu)\\
            & =0_{g}( f\circ \mu \circ i_{g}^{1})+ Y_{e}( f\circ \mu \circ i_{g}^{2}) \\
            & = Y_{e}( f \circ \lambda_{g})
        \end{align*}
        which proves the smoothness.

        2. Let $ X,Y  $ left-invariant vector fields on $ G $. We must show that $ [X,Y] $  is a left-invariant vector field. \begin{align*}
            d \lambda_{g}([X,Y]_{e})f & = [X,Y]_{g}f \\
            & = [X,Y]_{e}( f\circ \lambda_{g}) \\
            & = X_{e}(Y(f\circ \lambda_{g})) - Y_{e}(X(f \circ \lambda_{g})) \\
            & = X_{e}(d \lambda_{g}(Yf)) - Y_{e}(d \lambda_{g}(Yf)) 
        \end{align*}
    \end{proof}

    \section{18 Jan 2023}
    \vspace{2pt}
    \begin{lemma}
        Suppose $ \psi : M \to N  $ is a smooth map. Let $ X_{1},X_{2} $ be vector fields on $ M $, $ Y_{1},Y_{2} $ be vector fields on $ N $ such that $ X_{i} $ is $ \psi $-related to $ Y_{i} $. Then $ [X_{1},X_{2}]  $ is $ \psi $-related to $ [Y_{1},Y_{2}] $.
    \end{lemma}
    \begin{proof}
        Notice that \begin{align*}
            d \psi [X_{1},X_{2}](f) & = [X_{1},X_{2}]( f\circ \psi) \\
            & = X_{1}(X_{2} f \circ \psi) - X_{2}(X_{1} f\circ \psi) \\
            & = X_{1}( d \psi X_{2}f) -X_{2}(Y_{1} f \circ \psi) \\
            & = X_{1}(Y_{2} f \circ \psi) - X_{2}(Y_{1} f\circ \psi) \\
            & = d \psi X_{1}(Y_{2}f)- d \psi X_{2}(Y_{1}f) \\
            & = Y_{1}Y_{2} f\circ \psi - Y_{2}Y_{1}f \circ \psi \\
            & = [Y_{1},Y_{2}](f)\circ \psi
        \end{align*}
    \end{proof}
    This lemma proves that the set of left-invariant vector fields forms a Lie algebra. 

    Consider the Lie group $ \operatorname{GL}_{n}(\R) $. We want to verify the Lie algebra structure on $ \mathfrak{gl}_{n}(\R)  = M_{n}(\R)$ with the isomorphism \begin{align*}
        Lie(\operatorname{GL}_{n}(\R)) & \to \mathfrak{gl}_{n}(\R)\\ 
        X \xmapsto[]{\beta} X_{e}
    \end{align*}

    \begin{lemma}
        
        \[ \beta ([X,Y]) = [\beta(X),\beta(Y)] \]
        
    \end{lemma}
    \begin{proof}
        Evaluating the bracket on coordinate function $ x_{ij} $. \begin{align}
            [X,Y]_{e}(x_{ij}) = X_{e}(Y(x_{ij})) - Y_{e}(X(x_{ij})) \label{1}
        \end{align}

        Now \begin{align*}
            Y(x_{ij})(g) & = d \lambda_{g} Y_{e}(x_{ij}) \\
            & = Y_{e}(x_{ij} \circ \lambda_{g}) \\
            & = \sum_{k}^{} x_{ik}(g)Y_{e}(x_{kj}) 
        \end{align*}
        Considering the above as function of $ g $ and substituting this in \cref{1} we get \begin{align*}
            [X,Y]_{e}(x_{ij}) & = X_{e}Y_{e}(x_{ij}) - Y_{e}X_{e}(x_{ij}) \\
            & = \sum_{k} \{X_{e}(x_{ik})Y_{e}(x_{kj}) -  Y_{e}(x_{ik})X_{e}(x_{kj})\} \\
            & = [X_{e},Y_{e}](x_{ij})
        \end{align*}
    \end{proof}
    \begin{defn}
        A \textbf{Lie subgroup} $ H $ of a Lie group $ G $ is a submanifold $ H \xrightarrow{\alpha} G $ where $ \alpha $ is smooth and a group homomorphism. 
    \end{defn}
    We say that $ H $ is closed Lie subgroup if it is Lie subgroup such that $ H \to \alpha(H) $ is a diffeomorphism.
    \begin{example}
        Consider the map $ \R \to S^{1} \times S^{1} $ given by 
        \[ t \mapsto (e^{2 \pi it}, e^{2 \pi i \sqrt{2}t}) \]
        The image is a Lie subgroup of $ S^{1} \times S^{1} $ but it is not a closed Lie subgroup. It is also known as ``Skew-line'' in the torus. 
    \end{example}
    \begin{defn}
        Let $ \mathfrak{g}, \mathfrak{h} $ be Lie algebras and $ f: \mathfrak{g} \to \mathfrak{h} $ be a vector space homomorphism. Then we say that $ f $ is a Lie algebra homomorphism if 
    \[ f([X,Y]) = [f(X),f(Y)] \]
    \end{defn}
    \begin{thm}
        Suppose that $ \psi : G \to H $ is a Lie group homomorphism. Let $ X $ be a left-invariant vector field on $ G $. Extend $ d \psi(X_{e}) = Y_{e} \in T_{e}H $ to a left-invariant vector field $ Y $ on $ H $. Then $ X $ and $ Y $ are $ \psi $-related. This implies $ d \psi_{e} : \mathfrak{g} \to \mathfrak{h} $ is a Lie algebra homomorphism.
    \end{thm}
    \begin{proof}
        Consider the commutative diagram 

        % https://q.uiver.app/?q=WzAsNSxbMCwwLCJURyJdLFsyLDAsIlRIIl0sWzAsMiwiRyJdLFsyLDIsIkgiXSxbMSwxXSxbMiwzLCJcXHBzaSJdLFsxLDNdLFswLDJdLFswLDEsImQgXFxwc2kiXSxbMiwwLCJYIiwwLHsiY3VydmUiOi0yfV0sWzMsMSwiWSIsMix7ImN1cnZlIjoyfV1d
\[\begin{tikzcd}
	TG && TH \\
	& {} \\
	G && H
	\arrow["\psi", from=3-1, to=3-3]
	\arrow[from=1-3, to=3-3]
	\arrow[from=1-1, to=3-1]
	\arrow["{d \psi}", from=1-1, to=1-3]
	\arrow["X", curve={height=-12pt}, from=3-1, to=1-1]
	\arrow["Y"', curve={height=12pt}, from=3-3, to=1-3]
\end{tikzcd}\]

        We want to show that $ Y \circ \psi = d \psi \circ Y $. Now \begin{align*}
            \lambda_{\psi(g)} \circ \psi & = \psi \circ \lambda_{g}  
        \end{align*}
        so \begin{align*}
            Y_{\psi(g)} & = d \lambda_{\psi(g)}Y_{e} \\
            & = d \lambda_{\psi(g)} d \psi X_{e} \\
            & = d(\lambda_{\psi(g)} \circ \psi)(X_{e}) \\
            & = d (\psi \circ \lambda_{g}) (X_{e}) \\
            & = d \psi d \lambda_{g}(X_{e}) \\
            & = d \psi X_{g}
        \end{align*}
    \end{proof}

    \begin{thm}\label{psiphi}
        Let $ G,H $ be Lie groups with $ G $ connected. Let 
        \[ \phi,\psi : G \to H \] be homomorphism of Lie groups such that 
        \[ d \phi = d \psi : T_{e}G \to T_{e}H \]
        Then $ \phi=\psi $.
        
    \end{thm}

    \section{23 Jan 2023}

    I missed the class. Regardless here are some definitions from Warner covered on this day.

    \begin{defn}
        Let $ M $ be a smooth $ d$-dimensional manifold. For any integer $ 1 \le c \le d $, a \textbf{$ c $-dimensional distribution} $ \mathscr{D} $ on manifold is a choice of $ c $-dimensional subspace $ \mathscr{D}_{p} \subset T_{p}M $. $ \mathscr{D} $ is smooth if for each $ p \in M $ there is an open neighborhood $ U $ of $ p $ and there are $ c $  smooth vector fields $ X_{1}, \ldots , X_{c} $  on $ U $ which span $ \mathscr{D}_{m} $ for each $ p \in U $.

        We say $ \mathscr{D} $ is \textbf{involutive}  if $ [X,Y] \in \mathscr{D} $ whenever $ X,Y \in \mathscr{D} $.
    \end{defn}

    \begin{defn}
        A submanifold $ (N, \phi) $ of $ M$ is an integral manifold of a distribution $ \mathscr{D} $ if 
        \[ d \phi(N_{p}) = \mathscr{D}_{\phi(p)} \]
    \end{defn}

    Suppose there exists an integral manifold $ N $ for a distribution $ \mathscr{D} $, then for the points on $ N $ the distribution $ \mathscr{D} $ is necessarily involutive. Frobenius theorem states that it is sufficient condition for a distribution to be integral. 

    \begin{thm}
        \textbf{(Frobenius)}  Let $ \mathscr{D} $ be a $ c $-dimensional involutive smooth distribution on $ M $. Then there exists an integral manifold of $ \mathscr{D} $ passing through each point of $ M $.
    \end{thm}

    \subsection*{Differential Ideals}

    Let $ E^{*}(M)  = \bigoplus_{i=0}^{\infty} E^{i}(M) $ denote the graded algebra of smooth differential forms over manifold $ M $. 
    \begin{defn}
        Let $ \mathscr{D} $ be a smooth $ p$-dimensional distribution on $ M $. A $ q $-form $ \omega $ is said to \textbf{annihilate} $ \mathscr{D} $ if for each $ x \in M $ 
        \[ \omega_{x}\left(v_{1}, \ldots , v_{q}\right) = 0 \quad\text{  whenever  }v_{1}, \ldots, v_{q} \in \mathscr{D}_{x}\]
        
    \end{defn}

    A form $ \omega \in E^{*}(M)  $ is said to annihilate $ \mathscr{D} $ if each of the homogenous components of $ \omega $ annihilate $ \mathscr{D} $. Define 
    \[ \mathscr{I}( \mathscr{D})  \define \{ \omega \in E^{*}(M): \omega \text{ annihilates }\mathscr{D}\}\]
    

    \begin{defn}
        An ideal $ \mathscr{I} \in E^{*}(M) $ is called a \textbf{differential ideal} if it is closed under exterior differentiation; i.e. 
        \[ d (\mathscr{I}) \subset \mathscr{I} .\]
    \end{defn}
    \begin{thm}
        A smooth distribution $ \mathscr{D} $ on $ M $ is involutive if and only if the ideal $ \mathscr{I}( \mathscr{D}) $ is a differential ideal.
    \end{thm}
    \section{25 Jan 2023}
    \vspace{0.5cm}
    \begin{thm}
        If $ \phi :H \to G  $ is a homomorphism of Lie groups and if $ \omega  $ is a left-invariant differential form on $ G $, then $ \phi^{*}(\omega ) $ is again a left-invariant form on $ H $.
    \end{thm}

    Suppose that $ \phi : H \to G $ is a homomorphism of Lie groups. Let $ \omega_{1}, \ldots, \omega_{d} $ be a basis for $ E^{1}_{\text{inv}}(G) $. Then 
    \[ \mathcal{I}_{\phi} = \left< \{\pi_{1}^{*}\phi^{*}(\omega_{j}) - \pi_{2}^{*}(\omega_{j})\} : 1 \le j \le d \right> \]
    is a left-invariant differential ideal of $ H \times G $.

    \begin{lemma}
        Suppose $ X_{1}, \ldots, X_{d} $ is a basis of $ \mathfrak{g} $ dual to $ \omega_{1}, \ldots, \omega_{d} $. Suppose the Lie bracket is given by 
        \[ [X_{i},X_{j}]  = \sum_{}^{} c_{ij}^{k}X_{k} \]
        Then the $ C^{\infty} $ functions $ c_{ij}^{k} $ are constant. Further, \begin{align*}
            d \omega_{i} = -c_{kj}^{i} \omega_{k} \wedge \omega_{j}
        \end{align*}
    \end{lemma}
    \begin{proof}
        Notice that \begin{align*}
            d \omega_{k}(X_{i},X_{j}) & = - \omega_{k}([X_{i},X_{j}]) \\
            & = - c_{ij}^{k} 
        \end{align*}
        which is a constant because a left-invariant $ 1 $-form evaluated on a left-invariant vector field is a constant.  
    \end{proof}
    \begin{remark}
        The $ c_{ij}^{k} $ are called the structural constants of $ G $ with respect to the basis $ \{X_{i}\} $ of $ \mathfrak{g} $.
    \end{remark}
    \vspace{1cm}
    \begin{proof}
        \cref{psiphi}. Notice that $ \mathcal{I}_{\psi} = \mathcal{I}_{\phi} $ since $ d \phi = d \psi $ and these are invariant differential ideals; hence integral manifolds of $ \mathcal{I}_{\phi} $ and $ \mathcal{I}_{\psi} $ passing through $ (e,e) $ are the same. Thus, $ \phi = \psi $.
    \end{proof}

    \begin{lemma}
        Suppose $ G $ is any Hausdorff topological group which is connected. Suppose $ e \in U \subset G $ is any open set. Then 
        \[ G = \bigcup_{n \ge 1} U^{n}\]
        where $ U^{n} = \{x_{1} \cdots x_{n} |  x_{i} \in U\} $
    \end{lemma}
    \begin{proof}
        Since $ e \in U$ is open, $ U^{-1}  = \{x^{-1}| x \in U\}$ is also an open neighborhood of $ e $. Let $ V = U \cap U^{-1} $. Note that 
        \[ H \define \bigcup_{n \ge 1}V^{n} \]
        is a subgroup of $ G $, and it is open. Since the cosets $ gH $ are also open it follows that $ G =  \cup_{g}H $ being connected must be $ H $.
        
    \end{proof}

    \begin{thm}
        Let $ G $ be a Lie group and $ \mathfrak{h} \subset \mathfrak{g} $ be a Lie subalgebra of $ \mathfrak{g} $. Then there exists connected Lie subgroup $ H $ of $ G $ such that $ T_{e}H = \mathfrak{h} $.
    \end{thm}
    \begin{proof}
        Consider the distribution $ \mathscr{D} $ defined as 
        \[ \mathscr{D}_{g} = \{X_{g}| X \in \mathfrak{h} \} \]
        on $ G $. Suppose $ X_{1}, \ldots, X_{c} $ is a basis of $ \mathfrak{h} $. Then $ \mathscr{D} $ is generated by $ X_{1}, \ldots,X_{c} $ and $ \mathscr{D}  $ is involutive.
    \end{proof}
    \begin{corollary}
        (a) There is a one-to-one correspondence between connected Lie subgroups of $ G $ and Lie subalgebras of $ \mathfrak{g} $.

        (b) Suppose $ (H,i) \leftrightarrow \mathfrak{h} \subset \mathfrak{g} $. Then $ (H,i) $ is an embedded manifold if and only if $ H $ is closed.
    \end{corollary}

    \begin{thm}
        Suppose that $ A \subset G$ is an abstract subgroup of $ G $ and if $ A  $ has a manifold structure such that $ (A,i) \to G $ is a submanifold. Then the manifold structure is unique, $ A $ is a Lie group and hence $ (A,i) $ is a Lie subgroup of $ G $.
    \end{thm}
    \begin{thm}\textbf{(Ad\'o)} 
        Suppose that $ \mathfrak{g} $ is a finite dimensional Lie algebra. Then $ \mathfrak{g} $ can be realized as a  subalgebra of $ \mathfrak{gl}(n, \R) $.
    \end{thm}
    
    Given any connected Lie group $ G $, it has a universal cover $ \tilde{G} \xrightarrow{\pi} G $. Choose $ \tilde{e} \in \pi^{-1}(e) \in \tilde{G} $ such that the following diagram 

    % https://q.uiver.app/?q=WzAsNCxbMCwwLCJcXHRpbGRle0d9XFx0aW1lc1xcdGlsZGV7R30iXSxbMiwwLCJcXHRpbGRle0d9Il0sWzIsMiwiRyJdLFswLDIsIkcgXFx0aW1lcyBHIl0sWzAsMV0sWzMsMl0sWzEsMiwiXFxwaSJdLFswLDMsIlxccGkgXFx0aW1lcyBcXHBpIiwyXV0=
\[\begin{tikzcd}[ampersand replacement=\&]
	{\tilde{G}\times\tilde{G}} \&\& {\tilde{G}} \\
	\\
	{G \times G} \&\& G
	\arrow[from=1-1, to=1-3]
	\arrow[from=3-1, to=3-3]
	\arrow["\pi", from=1-3, to=3-3]
	\arrow["{\pi \times \pi}"', from=1-1, to=3-1]
\end{tikzcd}\] commutes.

\section{30 Jan 2023}
\vspace{0.5cm}
\begin{lemma}
    Suppose that $ G $ is a connected Lie group. Then $ \pi_{1}(G) $ is abelian.
\end{lemma}
\begin{proof}
    Suppose $ \sigma, \tau : I \to G $ be two loops. Define $ \sigma \cdot \tau $ by 
    \[ (\sigma \cdot \tau)(s) = \sigma (s) \cdot \tau (s) \]
    Then we have 
    \[ \sigma * \tau \cong \sigma \cdot \tau \]
    where $ * $ denote the product in the fundamental group $ \pi_{1}(G) $ (given by concatenation) and $ \cong $ denotes equivalent in homotopy. Also, 
    \[ \sigma \cdot \tau \cdot \sigma^{-1} \cong \tau\]
    which implies $  \sigma \tau \cong \tau \cdot \sigma $
\end{proof}
\begin{thm}
    Suppose that $ G $ and $ H $ are Lie groups with Lie algebras $ \mathfrak{g} $ and $ \mathfrak{h} $ with $ G $ simply connected. Let $ \tilde{\phi} : \mathfrak{g} \to \mathfrak{h} $ be a Lie algebra homomorphism. Then there exists a Lie group homomorphism 
    \[ \phi : G \to H \]
    such that $ d \phi_{e} : T_{e}(G) = \mathfrak{g}\to \mathfrak{h} = T_{e}H $ is equal to $ \tilde{\phi} $.
\end{thm}
\begin{proof}
    [Idea] Let $ \{\omega_{i}\} $ be a basis for invariant differential forms in $ E^{1}(H) $. Let $ \mathscr{I} $ be the ideal generated by $ \{\pi_{1}^{*}\tilde{\phi}^{*}(\omega_{j}) - \pi_{2}^{*}(\omega_{j})| 1\le j \le d\} $. Then $ \mathscr{I} $ is an invariant differential ideal of $ G \times H $, so it comes from vanishing of an integrable submanifold of $ G \times H $ passing through $ (e,e) $. 

    Then $ M $ is a Lie subgroup of $ G \times H $ and $ M \xrightarrow{p} G $ obtained by restriction of $ \pi_{1} $ is a group homomorphism and also a local diffeomorphism. So $ p: M \to G $ is a covering projection but $ G $ is simply connected so $ p $ is a diffeomorphism 
    \[ G \xrightarrow{p^{-1}} M \hookrightarrow G \times H \to H. \]
\end{proof}

\begin{corollary}
    \begin{enumerate}
        \item Suppose $ \mathfrak{ g} \cong \mathfrak{h} $ as Lie algebras and $ G $ and $ H $ are simply connected. Then $ G \cong H $ as Lie groups.
        \item There exists a one-to-one correspondence between (finite dimensional) Lie algebras and simply connected Lie groups.
        \item The differential structure of a Lie group is determined by its Lie algebra.
    \end{enumerate}
\end{corollary}

If $ G $ is a topological group which is locally Euclidean, does it support a Lie group structure? The answer is yes but the proof is quite difficult.

\subsection*{Exponential map}  
Let $ X  $ be a left-invariance vector field on $ G $. We have a Lie algebra homomorphism \begin{align*}
    \text{Lie}(\R) \cong \R & \to \mathfrak{g} \\
    c \frac{d}{dt} & \to cX
\end{align*}
This yields a Lie group homomorphism \begin{align*}
    \R & \xrightarrow{\exp_{X}}  G  \\
    x & \mapsto \exp_{X}(x)
\end{align*}
then $ d \exp_{X}(c \frac{d}{dt}) = cX $. The map \begin{align*}
    \mathfrak{g} & \xrightarrow{\exp} G \\
    X & \mapsto \exp_{X}(1) 
\end{align*}
is called the \textbf{exponential map}. 
\begin{thm}\label{expo}
    Let $ X \in \text{Lie}(G) $. Then \begin{enumerate}
        \item $ \exp(tX) = \exp_{X}(t) $ 
        \item $ \exp(t_{1}X_{1}+t_{2}X) = \exp(t_{1}X) \cdot \exp(t_{2}X) $ 
        \item $ \exp(-tX) = ( \exp(tX))^{-1} $ 
        \item $ \exp : \mathfrak{g} \to G $  is smooth and $ d \exp : T_{0} \mathfrak{g} \to T_{e}G = \mathfrak{g} $ is the identity map 
        \item $ \lambda_{g} \circ \exp_{X} : \R \to G  $ is the unique integral curve of $ X $ which is based at $ g $.
        \item The left-invariant vector fields are complete, i.e. their integral curves exist for all time.
        \item The one-parameter group of diffeomorphism $ \psi_{X,t}  $ for $ t \in \R $ is given by 
        \[ \psi_{X,t} = \rho_{exp_{X}(t)} \]
        where $ \rho_{g} $ denote right-multiplication by $ g $.
    \end{enumerate}
\end{thm}
\begin{thm}
    Suppose $ \psi : H \to G $ is a Lie group homomorphism. Then 
    \[\begin{tikzcd}[ampersand replacement=\&]
        { \mathfrak{h}} \&\& { \mathfrak{g}} \\
        \\
        {H} \&\& G
        \arrow["d \psi", from=1-1, to=1-3]
        \arrow["\psi", from=3-1, to=3-3]
        \arrow["\exp", from=1-3, to=3-3]
        \arrow["{ \exp}"', from=1-1, to=3-1]
    \end{tikzcd}\] commutes.
\end{thm}
[DO THIS COMMUTATIVE DIAGRAM.]

\section{1 Feb 2023}
\vspace{0.5cm}
\begin{thm}
    Suppose that $ \mathfrak{h} \subset \mathfrak{g} $ is a Lie subalgebra where $ \text{Lie}(G) $. Let $ A \subset G $ an abstract subgroup such that there exists a neighbourhood $ 0 \in V \subset \mathfrak{g} $ such that 
    \[ \exp(V \cap \mathfrak{h}) = U \cap H \]
    for some neighborhood $ e \in U \subset G $. Then $ H $ has a unique manifold structure such that $ (H, i) \hookrightarrow G $ is an embedded submanifold of $ G $ and $ H $ is closed in subset topology.
\end{thm}
\begin{remark}
    Lines with irrational slope in torus doesn't satisfy the hypothesis.
\end{remark}

\subsection*{Matrix exponentiation} 

Recall that $ \mathfrak{gl}(n,\R) $ denotes the Lie algebra of $  n \times n $ matrices over $ \R $ and similarly for $ \mathfrak{gl}(n,\C) $. 
\begin{defn}
    Define a map \begin{align*}
        \mathfrak{gl}(n,\C) &\to \operatorname{GL}(n,\C) \\
        A &  \mapsto e^{A} = \sum_{k=0}^{\infty} \frac{A^{k}}{k!} 
    \end{align*}
\end{defn}
It can be proved that the series is convergent with sup norm and further we have a lemma
\begin{lemma}
    If $ AB=BA $ then 
    
    \[ e^{A+B} = e^{A}e^{B} \]
\end{lemma}
which can be used to prove that $ e^{A} \in \operatorname{GL}(n,\C) $, so the definition makes sense.

Fix $ A $  and consider the function \begin{align*}
    \R \ni t \mapsto e^{tA} \in \operatorname{GL}(n,\C)
\end{align*} then its derivative is 
\[ \frac{d}{dt}\bigg|_{t=0}e^{tA} = A \]
because we can differentiate term by term in uniform convergence. This confirms \cref{expo} 4th part.

The left-invariant vector field given by $ A \in \mathfrak{gl}(n,\C) $ is just  multiplication by $ A $ on the right. Thus, $ t \mapsto e^{tA} $ is the integral curve associated to the vector field $ A \in \mathfrak{gl}(n,C) $ based at $ I $. Hence, this is the exponential map in the cases of $ \operatorname{GL}(n,\C) $.

\begin{thm}
    The exponential map $ \exp : \mathfrak{g} \to G $ is smooth.
\end{thm}
\begin{proof}
    Let $ X \in \mathfrak{g} $ and consider the map \begin{align*}
        V : G \times \fg & \to TG \times \fg \\
        (g,X) & \mapsto (X_{g},0)
    \end{align*}
    then $ V $ is smooth. Also, $ V $ is left-invariant on $ G \times \fg $. Consider the integral curve $ \gamma  $ based at $ (g,X) $ of $ V $. Then 
    \[ \gamma_{V}(t) = (g \exp_{X}(t),X) \]
    because of left invariance so 
    \[ \gamma_{V}(1) = (g \exp(X),X) \]
    \begin{align*}
        G \times \fg & \xrightarrow{\gamma_{V}(1)} G \times \fg \xrightarrow{\pi} G \\
        (e,X) & \mapsto \gamma_{V}(1) \to \exp(X)
    \end{align*}
\end{proof}

\section{6 Feb 2023}

Note that exponential map commutes with Lie group homomorphisms. Using Ado's theorem we get that for any Lie group

% https://q.uiver.app/?q=WzAsNCxbMCwwLCJHIl0sWzIsMCwiR0wobixcXG1hdGhiYntDfSkiXSxbMiwyLCJcXG1hdGhmcmFre2dsfShuLFxcbWF0aGJie0N9KSJdLFswLDIsIlxcbWF0aGZyYWt7Z30iXSxbMywyXSxbMiwxLCJcXGV4cCJdLFszLDAsIlxcZXhwIiwyXSxbMCwxLCJcXHBzaSJdXQ==
\[\begin{tikzcd}[ampersand replacement=\&]
	G \&\& {GL(n,\mathbb{C})} \\
	\\
	{\mathfrak{g}} \&\& {\mathfrak{gl}(n,\mathbb{C})}
	\arrow[from=3-1, to=3-3]
	\arrow["\exp", from=3-3, to=1-3]
	\arrow["\exp"', from=3-1, to=1-1]
	\arrow["\psi", from=1-1, to=1-3]
\end{tikzcd}\]

Consider the Lie group $ \operatorname{SL}(n,\C) = \{X \in \operatorname{GL}(n,\C) | \det(X) = 1\} $, for any $ A \in \mathfrak{gl}(n,\C) $ upper triangular with diagonal entries $ \lambda_{1}, \ldots, \lambda_{n} $ then 
\[ \det(e^{A})  = e^{\lambda_{1}+ \ldots + \lambda_{n}} = e^{\text{tr}(A)} \]
Now $ \mathfrak{sl}(n,\C) = \{A \in \mathfrak{gl}(n,\C) |  \text{tr}(A) = 0\} $, then $ \mathfrak{sl}(n,\C) $ is a Lie subalgebra of $ \mathfrak{gl}(n,\C) $ and exponential maps $ \mathfrak{sl}(n,\C) $ to the Lie subgroup $ \operatorname{SL}(n,\C) $. As $ \operatorname{SL}(n,\C) $ is a closed subgroup of $ \operatorname{GL}(n,\C) $ and dimension $ 2(n^{2}-1) $. Using Theorem 8.1 on an appropriate neighborhood we can complete the proof. 

\begin{align*}
    \text{Lie subgroup} &  && \text{Lie subalgebra } \mathfrak{gl}(n,\C) \\
    U(n) & \longleftrightarrow && u(n) = \text{skew-Hermitian matrices} \\
    \operatorname{SU}(n) & \longleftrightarrow && su(n) = \text{ skew-Hermitian + trace}=0
\end{align*}
Prove the above given correspondence using this lemma (TO DO).
\begin{lemma}
    Suppose that $ P \in \operatorname{GL}(n,\C) $ and $ A \in \mathfrak{gl}(n,\C) $, then 
    \[ Pe^{A}P^{-1} = e^{PAP^{-1}}. \]
     
\end{lemma}

\begin{thm}
    [Baker-Campbell-Hausdorff formula] Let $ \fg $ be a Lie algebra corresponding to a connected Lie group $ G $. Then in a neighborhood $ U $ of the identity, the multiplication $ U \times U \to G $ is completely determined by Lie algebra structure of $ \fg $. There is a formula for $ Z = Z(X,Y) $, $ X,Y \in V \subset \fg $, where $ e^{X}\cdot e^{Y} = e^{Z}$ 
    \[ Z =  X+Y + \frac{1}{2}[X,Y]+ \frac{1}{12}[X,[X,Y]]+ \ldots \]
\end{thm}
Consider \begin{align*}
    e^{tX} \cdot e^{tY} & = \left( \sum_{}^{} \frac{t^{k}X^{k}}{k!} \right) \left( \sum_{}^{} \frac{t^{l}Y^{l}}{l!} \right) \\
    & = \sum_{m \ge 0}^{} \left( \sum_{k+l=m}^{} \frac{X^{k}Y^{l}}{k!l!}  \right)t^{m} 
\end{align*}
Suppose $ Z = tZ_{1}+ t^{2}Z_{2} +t^{3}Z_{3} \ldots $, then \begin{align*}
    e^{Z} & = 1+(tZ_{1}+ t^{2}Z_{2}+ \ldots) + \frac{\left( tZ_{1}+ t^{2}Z_{2}+  \right)}{2!} + \ldots \\
    & = 1+ t(Z_{1})+ t^{2}\left(Z_{2}+ \frac{Z_{1}^{2}}{2!}\right)
\end{align*}
So we get $ Z_{1} = X+Y $, \begin{align*}
    \frac{X^{2}}{2!}+ XY+ \frac{Y^{2}}{2!} & = Z_{2}+ \frac{Z_{1}^{2}}{2!} \\
    & = Z_{2} + \frac{1}{2}\left( X^{2}+ XY+YX+Y^{2} \right)
\end{align*}
so $ Z_{2} = XY - \frac{1}{2}\left( XY+YX \right)  = \frac{1}{2}\left( XY-YX \right) = \frac{1}{2}[X,Y]$

\begin{thm}
    Suppose that $ \psi : R \to G $ is a continuous homomorphism. The $ \psi $ is smooth.
\end{thm}
\begin{proof}
    It is enough to show that $ \psi $ is smooth at $ 0 $. Let $ U $ be a star-like neighborhood of $ 0 \in \fg $ such that $ \exp|_{U}  :U \to G $ is a diffeomorphism onto $ \exp(U) $. Let $ U' = \{\frac{X}{2} | X \in U\} $. Choose $ Y \in U' $ and let $ \psi(t_{0}) = \exp(Y) $. Choose $ t_{0} > 0  $ such that 
    \[ \psi([-t_{0},t_{0}]) \subset \exp(U') \]
    Let $ n \ge 2 $, and suppose that $ X \in U'
     $ such that $ \exp(X) = \psi(  \frac{t_{0}}{n}) $. Claim $ nX = Y $
\end{proof}
\section{6 Feb}
\section{8 Feb}
\section{13 Feb}

\vspace{0.5cm}
\begin{defn}
    Let $ \mathfrak{a} \in \fg $ be a Lie subalgebra of a Lie algebra $ \fg $. We say that $ \ag $ is an \textbf{ideal} in $ \fg $ if $ [X,Y] \in \ag $ for all $ X \in \fg $ and $ Y \in \ag $. 
\end{defn}


\begin{thm}
    Suppose $ A \subset g$ is a connected Lie subgroup of a connected Lie group $ G $. Then $ A $ is normal in $ G $ if and only if $ \ag = \text{Lie}(A) $ is an ideal in $ \fg $.
\end{thm}
\begin{proof}
    Suppose that $ \ag  \subset \fg $ is an ideal. Let $ g \in G $, $ h \in A $. We must show that $ ghg^{-1} \in A $, to do this it is enough to show this for $ g $ in a neighborhood of $ e $ and $ h $ in a neighborhood of $ e $ in $ A $. So we may write $ g = \exp X, h = \exp Y $  \begin{align*}
        ghg^{-1} & = \exp \circ \operatorname{Ad}_{g}(Y) \\
        & = \exp \operatorname{Ad}_{\exp(X)}(Y) \\
        & = \left( \exp \left( \exp(id_{X}) \right) \right) \\
        & = \exp \left( I + \operatorname{ad}_{X} + \frac{\operatorname{ad}_{X}^{2}}{2!} + \ldots \right)(Y) \\
        & = \exp\left(Y+ [X,Y]+ \frac{[X,[X,Y]]}{2!}+ \ldots \right) \in A
    \end{align*}

Now assume $ A $ is normal in $ G $. Let $ X \in \fg $, $ Y \in \ag $. Write $ g_{t} = \exp tX $. We know that \begin{align*}
   A \ni g_{t} ( \exp(sY))g_{t}^{-1} & = \exp( \operatorname{Ad}_{g_{t}}(sY)) \\
   & = \exp(s \operatorname{Ad}_{g_{t}}) \\
   & = \exp(s \exp \operatorname{ad}_{tX}(Y))
\end{align*}

This implies $ \exp \operatorname{ad}_{tX}(Y) \in \ag $ so $ Y+ t[X,Y] + \frac{t^{2}}{2!}[X,[X,Y]] + \dots $ and using \newline 
$ \dfrac{d}{dt}\bigg|_{t=0} \exp \operatorname{ad}_{tX}(Y) = [X,Y] \in \ag $.
\end{proof}

\begin{defn}
    The center of a Lie algebra $\fg $ is the vector space $ \mathfrak{z} = \mathfrak{z}(\fg) = \{X \in \fg \,  | \, [X,Y] = 0 \, \forall \, Y \in \fg\} $.
\end{defn}
\begin{remark}
    Note that $ \mathfrak{z} $ is trivial Lie subalgebra of $ \fg $.
\end{remark}

\begin{thm}
    Let $ Z = Z(G) $ be the center of $ G $. Then $ Z(G) = \ker ( \operatorname{Ad} : G \to \operatorname{GL}(\fg)) $.
\end{thm}
\begin{proof}
    If $ \fg \in Z(G) $, then $ i_{g}: G \to G = \operatorname{id}_{G} $ where $ i_{g} $ is the conjugation map. Taking the differential, this implies $ A_{g} : \fg  \to \fg$ is identity, hence $ g \in \ker( \operatorname{Ad}) $. 

    Suppose that $ g \in \ker( \operatorname{Ad}) $, %then $ g \in Z(G) $. It is enough to show that elements in a neighborhood of $ e $ in $ \ker( \operatorname{Ad}) $. 
    so $ \operatorname{Ad}_{g}(X) = X $. Let $ X \in \fg $ then \begin{align*}
        \exp tX & = \exp (t \operatorname{Ad}_{g}(X)) \\
        & = g \exp(tX) g^{-1}
    \end{align*}
    so $ g $ commutes with elements $ \exp(tX) $ in a neighborhood of $ e $, but that is enough since elements of the form $ \exp tX $ for any $ t \in \R, X \in \fg$ generate $ G $. Therefore, $ g \in Z(G) $. 
\end{proof}

\begin{proposition}
    If $ X,Y \in \fg $ are such that $ [X,Y] = 0 $. Then 
    \[ \exp(X+Y) = \exp(X) \exp(Y) .\]
\end{proposition}
\begin{proof}
    Let $ \ag = \R X+ \R Y $. Then $ \ag $ is abelian subalgebra of $ \fg $. Then the corresponding Lie subgroup $ A $ is abelian. Define $ \alpha : \R \to G $ such that 
    \[ \alpha(t) = \exp(tX) \exp(tY) \in A\]
    It follows that $ \alpha(s+t) = \alpha(s) \alpha(t) $ since $ A $ is abelian. Now $ \alpha (t) = \exp(tZ) $ for some $ Z \in \fg $ where $ Z = \dfrac{d}{dt}\bigg|_{t=0} \alpha(t)$. \begin{align*}
        \frac{d}{dt} \alpha(t) & = \frac{d}{dt}\bigg|_{t=0} \exp(tX)+ \frac{d}{dt}\bigg|_{t=0} \exp(tY) \\
        & = X_{e}+Y_{e} 
    \end{align*}
    So $ Z_{e} = X_{e}+ Y_{e} $ and $ \exp(tZ) = \exp(tX) \exp(tY) $ for all $ t \in \R $.
\end{proof}

\section{15 Feb}

\begin{motivation}
    We will try to look into automorphism group of Lie group now and the expectation is that it is a Lie group itself. %For example in case of an $ n $-torus, $ \T^{n} $ the automorphisms are in bijection with 
\end{motivation}

Let $ \psi : V \otimes V \to V $ be a linear map. Consider the sets 
\[ A_{\psi}(V) = \{ \alpha \in \operatorname{GL}(V)| (\alpha u, \alpha v) = \alpha ((u,v))\}, \] i.e. the diagram commutes 
% https://q.uiver.app/?q=WzAsNCxbMCwwLCJWIFxcb3RpbWVzIFYiXSxbMiwwLCJWIl0sWzIsMiwiViJdLFswLDIsIlYgXFxvdGltZXMgViJdLFszLDIsIlxccHNpIl0sWzAsMSwiXFxwc2kiXSxbMCwzLCJcXGFscGhhIFxcb3RpbWVzIFxcYWxwaGEiLDFdLFsxLDIsIlxcYWxwaGEiLDFdXQ==
\[\begin{tikzcd}[ampersand replacement=\&]
	{V \otimes V} \&\& V \\
	\\
	{V \otimes V} \&\& V
	\arrow["\psi", from=3-1, to=3-3]
	\arrow["\psi", from=1-1, to=1-3]
	\arrow["{\alpha \otimes \alpha}"{description}, from=1-1, to=3-1]
	\arrow["\alpha"{description}, from=1-3, to=3-3]
\end{tikzcd}\]

and 
\[ Dev_{\psi}(V)  = \{f \in \operatorname{End}(V) |  f(\psi(u,v)) = \psi(f(u),v)+ \psi(u, f(v))\} \]

\begin{proposition}
    \begin{enumerate}
        \item $ A_{\psi}(V) $ is a closed subgroup of $ \operatorname{GL}(V) $. 
        \item $ Dev_{\psi}(V) $ is a Lie subalgebra of $ \fg(V) $.
    \end{enumerate}
\end{proposition}
\begin{proof}
    TO DO
\end{proof}
\begin{thm}
    Lie algebra of $ A_{\psi}(V) $ equals $ Dev_{\psi}(V) $.
\end{thm}
\begin{proof}
    Let $ \ag = Lie(A_{\psi}(V)) \subset \fg(V) = \operatorname{End}(V)$. We must show that $ \ag = Dev_{\psi}(V) $. Suppose that $ f \in \ag $, then $ \exp(tf) \in A_{\psi}(V) $ for all $ t $. We need to show that 
    \[ f \circ \psi = \psi \circ (f \otimes 1 + 1 \otimes f) \]
    To do this, let $ u,v \in V $, then \begin{align*}
        \exp tf (u,v) & = (\exp tf (u), \exp tf (v)) \\
        & = (u,v)+ (t f(u), v)+ (u, t f(v)) + \text{higher powers of }t
    \end{align*}
    so \begin{align*}
        f(u,v) = \frac{d}{dt}\bigg|_{t=0} \exp tf (u,v) & = (f(u),v)+ (u, f(v))
    \end{align*}
    so $ f \in Dev_{\psi}(V) $. 

    Let $ f \in Dev_{\psi}(V)  $, we must show that \begin{align*}
        \exp (tf) (u,v) & = ( \exp (tf) u, \exp(tf) v) \\
        i.e \qquad \qquad \exp (tf) \circ \psi & = \psi \circ ( \exp (tf) \otimes \exp (tf)) \qquad \forall u,v \in V \text{ and } \forall t \in \R
    \end{align*}

    As $ f \in Dev_{\psi}(V) $, we have \begin{align*}
        f \circ \psi & = \psi \circ ( f \otimes 1+ 1 \otimes f) \\
        f^{2} \circ \psi & = f \circ f \circ \psi \\
        & = f \circ \psi \circ ( f \otimes 1 + 1 \otimes f) \\
        & = \psi \circ ( f \otimes 1 + 1 \otimes f)^{2} 
    \end{align*}

    By induction, 
    \[ f^{n} \circ \psi = \psi \circ (f \otimes 1+ 1 \otimes f) \]
    and $ f \otimes 1 , 1 \otimes f : V \otimes V \to V \otimes V$ commutes. It follows that \begin{align*}
        \exp (tf) \circ \psi & = \sum_{}^{} \left( \frac{t^{k}f^{k}}{k!} \circ \psi \right) \\
        & = \sum_{}^{} \frac{t^{k}}{k!} \psi \circ (f \otimes 1 + 1 \otimes f)^{k} \\
        & = \psi \circ \sum_{}^{} \frac{t^{k}}{k!}(f \otimes 1 + 1 \otimes f)^{k} \\
        & = \psi \circ \exp (tf \otimes 1+ 1 \otimes tf) \\
        & = \psi \circ (tf \otimes 1) \circ \exp (1 \otimes tf) \\
        & = \psi \circ \exp (tf \otimes tf) \\
        & = \psi ( \exp (tf) \otimes \exp (tf))  
    \end{align*}
\end{proof}

Let $ V = \fg = \operatorname{Lie}(G) $ and $ \psi = [ \cdot, \cdot] : \fg \otimes \fg \to \fg$ be the Lie bracket. Then 
\[ A_{\psi}(V) = \operatorname{Aut}_{Lie}( \fg ) \subset \operatorname{GL}(\fg) \] and 
\[ \operatorname{Der}_{\psi}(V) = \operatorname{Lie}( \operatorname{Aut}(\fg)) \]
by the theorem. Note that $ G \xrightarrow{ \operatorname{Ad}} \operatorname{GL}(\fg)$ factors through $ G \to \operatorname{Aut}_{ \operatorname{Lie}}(\fg) $ and $ \fg \xrightarrow{ \operatorname{ad}} \operatorname{Der}(\fg) $.   


Let $ V $ be a finite dimensional vector space. Consider a bilinear form 
\[ B : V \times V \to F, \]
equipped with a linear map 
\[ V \otimes V \to F \]

An element $ g \in \operatorname{GL}(V) $ is $ B$-invariant if 
\[ (u,v) = (gu,gv) \qquad \qquad \forall u,v \in V\]

An element $ f \in \operatorname{End}(V) $ is $ B $-invariant if 
\[ (fu,v)+ (u,fv) = 0 \]
Then $ O_{B}(V) = \{g \in \operatorname{GL}(V)| g \text{ is }B \text{-invariant}\} $ is a closed Lie subgroup of $ \operatorname{GL}(V) $ with  Lie algebra $ B $-invariant linear map endomorphisms of $ V $.

\begin{example}
    Take $ V = \R^{n} $ and $ B $ is the standard inner product. Then $ O_{B}(V) = O(n) $.
\end{example}
\section{1 March}
Missed
\section{6 March}
Missed
\section{8 March}
Missed
\section{13 March}

\subsection*{Fundamental group of Lie groups}

Reference - Hall (?)

\subsection*{Complexification}
Let $ V $ be a real vector space. Then the complexification is the vector space $ V \otimes_{\R} \C = V_{\C} $. If $ V $ is a Lie algebra, then $ V_{\C} $ is a Lie algebra where the bracket operates on $ V_{\C} $ is the $ \C $-linear extension of that on $ V $. It is given by \begin{align*}
    [X+iY,X'+iY'] & = [X,X']-[Y,Y'] + i \left( [X,Y']+ [X',Y] \right)
\end{align*}
for all $ X,Y,X',Y' \in V $. Suppose that $ V $ is a real Lie algebra and $ W $ is a complex Lie algebra. Suppose $ f:V \to W $ is a Lie algebra homomorphism where $ W $ is regarded as a $ \R $-Lie algebra. Then $ f $ extends to a unique complex Lie algebra homomorphism
\[ f_{\C} : V \otimes \C \to W  \]
Suppose that $ W = V+ iV $ as $ \C$ vector space  and where $ V \cap iV = 0 $ (internal direct sum). Then we say that $ V $ is a real form of $ W $. 

Suppose $ W $ is a complex Lie algebra and $ V $ is a real Lie subalgebra contained in $ W $ which is a real form of $ W $. Then 
\[ V_{\C} \equiv W \]
as $ \C $-Lie algebra.

Q. Given a Lie algebra, when is it the Lie algebra of a compact Lie group? 

A. Something about Killing form and non-degeneracy of complexified Lie algebra and semisimple Lie algebra.

\section{20 March}

Suppose that $ \psi : H \to G $ is a Lie algebra homomorphism into a connected $ \C $-Lie group $ G $. Then $ d \psi : \hg \to \fg $ extends to a complex Lie algebra homomorphism 
\[ \hg_{\C} \xrightarrow{d \psi \otimes \C} \fg .\]

\begin{defn}
    We say that $ \psi : H \to G $ is a complexification of $ H $ if for any complex Lie group $ L $ and any real Lie group homomorphism $ f : H \to L  $, there exists a unique complex Lie group homomorphism $ \phi : G \to L$ such that 
    \[ f = \phi \circ \psi .\]
\end{defn}
Also, \begin{defn}
    A homomorphism of Lie groups $ \psi : G \to L $ is a complex Lie group homomorphism if $ G,L $ are complex and $ d \psi : \fg \to \mathfrak{l} $ is a complex Lie algebra homomorphism.
\end{defn}

Idea : Like we can complexify a real Lie algebra, we would like to have a concept of complexification of a Lie group, but we may not be able to do so for all real Lie groups.

Given a complex Lie group $ G $, any $ (H,\psi) $ whose complexification is $ G $ will be called a real form. For a given complex Lie group there can be more than one real form. E.g. consider $ SU(n) \subset \operatorname{SL}(n,\C) $, and it can be proven that $ SU(n) $ is a real form of $ \operatorname{SL}(n,\C) $ (also called compact form since $ SU(n) $ is compact) by dimension analysis. 

Consider the diagram % https://q.uiver.app/?q=WzAsMyxbMCwwLCJTVShuKSJdLFsyLDAsIkwgKGNvbXBsZXggTGllIGdyb3VwKSJdLFswLDIsIlNMKG4sXFxtYXRoYmJ7Q30pIl0sWzAsMiwiXFxwc2kiLDAseyJzdHlsZSI6eyJ0YWlsIjp7Im5hbWUiOiJob29rIiwic2lkZSI6InRvcCJ9fX1dLFsyLDEsIlxcZXhpc3RzID8gXFxwaGkiXSxbMCwxXV0=
\[\begin{tikzcd}
	{SU(n)} && {L} \\
	\\
	{SL(n,\mathbb{C})}
	\arrow["\psi", hook, from=1-1, to=3-1]
	\arrow["{\exists ? \phi}", from=3-1, to=1-3]
	\arrow[from=1-1, to=1-3]
    \arrow["f", from=1-1, to=1-3]
\end{tikzcd}\]
where $ L $ is a complex Lie group. At the Lie algebra level \[\begin{tikzcd}
	{\mathfrak{su}(n)} && {L} \\
	\\
	{\mathfrak{sl}(n,\mathbb{C})}
	\arrow[hook, from=1-1, to=3-1]
	\arrow["\theta", from=3-1, to=1-3]
	\arrow["df", from=1-1, to=1-3]
\end{tikzcd}\]
where $ \mathfrak{sl}(n,\C) = \mathfrak{su}(n) + i \mathfrak{su}(n) $, $ \exists  \operatorname{SL}(n,\C) \xrightarrow{\phi} L$ such that $ d \phi = \theta $ since $ \operatorname{SL}(n,\C) $ is simply connected. Then $ \phi  $ restricts to $ f $ since $ d \phi|_{ \mathfrak{su}(n)} = d f $.

\begin{thm}
    Let $ K $ be a compact connected Lie group. Then there exists a complex Lie group $ K_{\C} $ and a Lie group homomorphism $ f : K \to K_{\C} $ such that \begin{enumerate}
        \item $ f_{*} : \pi_{1}(K) \to \pi_{1}(K_{\C}) $ is an isomorphism. 
        \item $ \operatorname{Lie}(K_{\C})  = \operatorname{Lie}(K) \otimes \C$. 
        \item $ K_{C} $ is the compactification of $ K $.
    \end{enumerate}
\end{thm}

\begin{thm}
    Suppose that $ G $ is a complex linear connected semisimple Lie group. Then any maximal compact Lie subgroup $ K \subset G  $ is a real form of $ G $.
\end{thm}

%\begin{Remark}
    %Complex Lie group are very well-behaved because of bi invariant Haar measure so it is important to form links between complex and real case. 
%\end{Remark}

\section{22 March}

Let $ \beta $ be a symmetric bilinear form on $ V $, where $ V $ is a finite-dimensional vector space over $ \R $ or $ \C $. Let $ Q $ be the associated quadratic form 
\[ Q : V \to \R \qquad \text{ or } \qquad Q : V \to \C\]
\[ Q(\lambda v) = \lambda^{2}v \]

We have $ Q(V) = \beta (v,v) $, and $ \beta(u,v) = \frac{Q(u+v) -Q(u)-Qv}{2}$. Suppose that $ (V, \beta) $, $ (V, \beta') $ are quadratic spaces. Then we say that $ (V, \beta), (V, \beta') $ are equivalent if there exists $ T : V \to V $ such that 
\[ \beta'(u,v) = \beta(Tu,TV)  \quad \forall u,v \in V\]
Suppose that $ v_{1}, \ldots, v_{n} $ is a basis for $ V $. Then the matrix of $ \beta $ is $ B = (\beta(v_{i},v_{j})) $.

Let $ B,B' $ be the matrices of $ \beta, \beta' $. Then $ (V, \beta), (V, \beta') $ are equivalent if there exists $ T \in M_{n}(F) $ such that 
\[ B = \, ^{t}TBT \]
where $ ^{t} $ denotes transpose. Now if $ x = (x_{1} \dots x_{n})^{t} $, $ y = (y_{1}, \dots, y_{n})^{t} $ are vectors in $ F^{n} \equiv V $, then 
\[ x^{t}By = \beta(x,y) \]
and 
\begin{align*}
    \beta'(x,y) & = \beta(Tx,Ty) \\
    & = x^{t}T^{t}BTy \\
    & = x^{t}B'y
\end{align*}
which proves the statement. Suppose $ E_{1} \subset E $, $ (E, \beta) $ is a quadratic space. Then $ (E_{1},\beta|_{E_{1}}) $ is a quadratic space. 
\[ E_{1}^{\perp} = \{x \in E : \beta(x,y) = 0 \, \forall y \in E_{1}\} \]

\begin{lemma}
    Suppose that $ E_{1} \subset E $ and $ (E, \beta|_{E_{1}}) $ is non-degenerate. Then 
    \[ E = E_{1} \oplus E_{1}^{\perp} = E_{1} \perp E_{1}^{\perp} \]
    If $ (E, \beta) $ is non-degenerate, then $ (E_{1}^{\perp}, \beta_{E_{1}^{\perp}}) $ is also non-degenerate. 
\end{lemma}
\begin{proof}
    TO DO. 
\end{proof}

Example - Consider the quad space $ (H, \beta) $ where $ H = \R^{2} $ and $ Q((x,y)) = x^{2}-y^{2} $. Then $ (H, \beta) \cong (H, \beta') $ where $ Q'((x,y)) = xy $. One can calculate that 
\[ B = \begin{bmatrix}
    1 && 0 \\
    0 && -1
\end{bmatrix} \]
and 
\[ B ' = \begin{bmatrix}
    0 && 1 \\
    1 && 0
\end{bmatrix}, \]
so the forms are non-degenerate and similar using transformation $ T = \frac{1}{\sqrt{2}}\begin{bmatrix}
    1 && 1 \\
    1 && -1
\end{bmatrix} $. Suppose $ (V, \beta) $ is non-singular, if $ \beta|_{E} \equiv 0 $, then $ \dim E \le \frac{1}{2}\dim V $. Further \begin{lemma}
    If $ (V, \beta) $ is non-singular then 
\[ V = V_{1} \oplus \dots \oplus V_{n} \] 
where each $ V_{i} $ is $ 1 $-dimensional and $ (V_{i}, \beta|_{V_{i}}) $ is non-degenerate, $ V_{i} \perp V_{j} $ if $ i \neq j $, i.e. there exists a basis of $ V $ with respect to the matrix $ B $ of $ \beta $ is diagonal.
\end{lemma}
\begin{proof}
    The proof is by induction on dimension. First suppose that $ v \in V $ is non-zero then choose $ V_{1} = Fv $ then 
    \[ V = V_{1} \oplus V_{1}^{\perp} \]
    and $ (V_{1}^{\perp}, \beta|_{V_{1}^{\perp}}) $ is non-degenerate. Apply induction to $ (V_{1}^{\perp}, \beta|_{V_{1}^{\perp}}) $. 

    Suppose $ \beta(v,v) = 0 $. Choose by non-degeneracy of $ \beta $ a vector $ V \in V $ such that $ \beta(u,v) = 0 $. Notice that $ \beta(u+v,u+v) = 2\beta(u,v) \neq 0$ which lands us in earlier case.
\end{proof}

Now suppose that $ (V, \beta) $ is arbitrary. Let $ V_{0} = \operatorname{rad}(\beta) = \{x \in V : \beta(x,y) = 0 \forall y \in V\} $
Consider the quotient $ (\frac{V}{V_{0}}, \overline{\beta}) $ with 
\[ \overline{\beta}(u+V_{0},v+V_{0}) = \beta(u,v) \]
and $ \operatorname{rad}(\beta) = 0 $, so $ (\frac{V}{V_{0}}, \overline{\beta}) $ is non-degenerate. Main theorem \begin{thm}
    Over $ \R $ any non-degenerate $ \beta $ is equivalent to the bilinear form with basis 
    \[ \begin{bmatrix}
        I_{k} && \\
        && - I_{l}
    \end{bmatrix} \]
 with $ k+l = n$. Moreover, $ k,l $ are uniquely determined by $ \beta $.
\end{thm}

\begin{defn}
    Let $ (V, Q) $ be a quadratic space. The \textbf{Clifford algebra} $ C(Q) $ associated to it is an algebra over $ F $ with a homomorphism $ \theta : V \to C(Q) $ such that
\begin{enumerate}
    \item $ \theta(x)^{2} = Q(x) $ 
    \item $ C(Q) $ is universal with respect to 1st property, i.e. if $ \psi : V \to A $ is any vector space homomorphism to an $ F $-algebra such that 
    \[ \psi(x)^{2} = Q(x) \]
    then there exists a unique algebra homomorphism $ f $ such that % https://q.uiver.app/?q=WzAsMyxbMCwwLCJDKFEpIl0sWzIsMCwiQSJdLFsxLDIsIlYiXSxbMiwwLCJcXHRoZXRhIl0sWzIsMSwiXFxwc2kiLDJdLFswLDEsImYiXV0=
    \[\begin{tikzcd}
        {C(Q)} && A \\
        \\
        & V
        \arrow["\theta", from=3-2, to=1-1]
        \arrow["\psi"', from=3-2, to=1-3]
        \arrow["f", from=1-1, to=1-3]
    \end{tikzcd}\]
    commutes. 
\end{enumerate}
\end{defn}
We can construct the Clifford algebra by 
\[ C(Q) = \frac{T(V)}{\left< x \otimes x - \psi(x) \right>} \]
where $ T(V) $ is the tensor algebra of $ V $. 

\begin{example}
    1. $ V = \R $, $ Q(x) = -x^{2} $ then 
    \[ T(V) =\R \oplus \R e_{1} \oplus \R( e_{1} \otimes e_{1}) \oplus \dots    \]
    and $ C(Q) = \R \oplus \R e_{1} $, $ e_{1}^{2} = -1 $ so $ C(Q) \cong \C $. 

    2. $ V = \R $, $ Q'(x) = x^{2} $, then 
    \[ C(Q') = \R \oplus \R e_{1} \]
    with $ e_{1}^{2} = 1 $ so it is the polynomial ring $ \frac{\R[x]}{(x^{2}-1)} $. 
    
\end{example}
\section{27 March}
Missed  
\section{29 March}
Missed
\section{3 April}
Missed
\section{5 April}

\begin{lemma}[Schur's lemma]
    Suppose that $ G $ is a compact Lie group. Let $ V_{0}, V_{1} $ be a finite dimensional irreducible representation over $\C $. Then any $ G $-homomorphism $ \psi : V_{0} \to V_{1} $ is either $ 0 $ or an isomorphism. Moreover, any $ G $-homomorphism $ V_{0} \to V_{0} $ is a scalar multiple of the identity.
\end{lemma}
\begin{proof}
    If $ V $ is any irreducible representation, then $ V $ is simple i.e. the only subrepresentation of $ V $ are $ 0 $ and $ V $. Now $ \operatorname{im}(\psi) \subset V_{1}$ is a subrepresentation. Assume $ \psi \neq 0 $. Then $ \operatorname{im}(\psi) = V_{1} $. 

    Also, $ \ker \psi \subset V_{0} $ is a subrepresentation. If $ \ker \psi = V_{0} $, then $ \psi = 0 $ therefore $ \ker \psi \neq V_{0} $ which implies $ \ker \psi = 0 $. Since $ V_{0} $ is irreducible, the map $ \psi $ is one-one hence $ \psi  $ is an isomorphism. 

    For the second part, suppose $ \phi : V_{0} \to V_{0} $ is a $ G $-homomorphism. Let $ \lambda $ be an eigenvalue of $ \phi $. Then $ (\lambda I - \phi) $ is singular and is a $ G $-homomorphism. By previous part we get $ \lambda I - \phi \equiv 0 $ or $ \phi = \lambda I $.
\end{proof}

\subsection*{Representation ring of $ G $}

Let $ [V] $ denote the isomorphism class of finite dimensional $ G $-representation $ V / \C $. Consider the free abelian group $ A $ with basis $ \{[V] : V \text{ is a }G\text{-representation}\} $. We consider the subgroup of elements of the form 
\[ S = \{[V_{0} \oplus V_{1}] - [V_{0}]- [V_{1}]  : V_{0}, V_{1} \text{ are } G-\text{representations}\} \]
then $ RG \define A / S $ is an abelian group. Further we can define multiplication by 
\[ [V] \cdot [W] = [V \otimes W] \]
Distributivity follows from $ (V_{1} \oplus V_{2}) \otimes W \cong (V_{1} \otimes W) \oplus (V_{2} \otimes W $

    Remark : Given two representations $ (V, \pi) $ and $ (W, \sigma) $ the tensor $ (V \otimes W, \rho ) $ is also $ G $-representation via 
    \[ \rho(g)(a \otimes b) = \pi(g)a \otimes \sigma(g)b \]
    i.e. $ g \cdot(a \otimes b)  = g a \otimes g b$. 

    This makes $ RG $ a ring generated by the classes of irreducible representations of $ G $. 

    \begin{example}
        Any irreducible representation of $ S^{1} $ is one-dimensional. Let 
        \begin{align*}
            \chi_{n} : S^{1} & \to U(1) = S^{1} \\
            z & \mapsto z^{n}
        \end{align*}
        If $ V_{n} = (\C, \chi_{n}) $, then $ V_{m} \otimes V_{n} = \C $ as a vector space. 
        \[ g(u_{1} \otimes u_{2}) = g u_{1} \otimes g u_{2} =g^{m}u_{1} \otimes g^{n}u_{2} = g^{m+n}u_{1} \otimes u_{2}\]
        Further calculations gives $ RS^{1} \cong \Z[\chi_{1}, \chi_{1}^{-1}] $
    \end{example}

    Let $ V $ be a $ G $-representation over $ \C $ endowed with a $ G $-invariant. Fix $ u,v \in V $, we have a function $ \psi_{\pi,u,v} : G \to \C $ given by 
    \[ \psi_{\pi,u,v}(g) = \left< \pi(g)u,v \right> .\]

    This is called a matrix coefficient of $ G $. Then $ \psi_{\pi,u,v} \in L^{2}(G) $. 
    \begin{remark}
        Matrix coefficients form a dense subset of $ L^{2}(G) $ but we will not prove it. 
    \end{remark}
    
    Given a representation $ (V, \pi) $ of $ G $, we have a function 
    \begin{align*}
        \chi_{\pi} : G & \to \C \\
        \chi_{\pi}(g) & = \operatorname{tr}(\pi(g)).
    \end{align*}
    This is called the characteristic function of $ V $. Properties \begin{enumerate}
        \item $ \chi_{\pi} = \chi_{\sigma} $ if $ \pi \cong \sigma $. 
        \item $ \chi_{\pi \oplus \sigma} = \chi_{\pi}+ \chi_{\sigma} $
        \item $ \chi_{\pi \otimes \sigma} = \chi_{\pi} \cdot \chi_{\sigma} $
    \end{enumerate}

    \begin{lemma}
        The characteristic function $ \chi_{\pi} $ is a matrix coefficient. 
    \end{lemma}
    \begin{proof}
        Let $ v_{1}, \dots , v_{n} $ be a Hermitian basis, i.e. $ \left< v_{i}, v_{j} \right> = \delta_{ij}$. Then 
        \[ \pi(g) = \left( \left< \pi(g)v_{i},v_{j} \right> \right)_{i,j} \]
        therefore 
        \[ \chi_{\pi}(g) = \sum_{i=1}^{n}\left< \pi(g)v_{i},v_{j} \right>  \]
        
        Now it is enough to show that sum of two matrix coefficients is again a matrix coefficient. Suppose $ \rho_{1}, \rho_{2} $ are $ G $-representation and $ u_{i},v_{i} \in V_{i} $, 
        \[ \psi_{\rho_{1},u_{1},v_{1}}(g)+ \psi_{\rho_{2},u_{2},v_{2}}(g) = \psi_{\rho_{1} \oplus \rho_{2}, (u_{1},u_{2}),(v_{1},v_{2})}(g) \]
        on $ V_{\rho_{1} \oplus \rho_{2}} = V_{\rho_{1}} \oplus V_{\rho_{2}} $.
    \end{proof}

    \begin{thm}[Schur orthogonality]
        If $ (V_{1}, \rho_{1}) $ and $ (V_{2},\rho_{2}) $ are irreducible representations over $ \C $ of a compact Lie group $ G $, then 
        \[ \left< \chi_{\rho_{1}}, \chi_{\rho_{2}} \right> = \begin{cases}
            0 \text{ if }V_{1} \neq V_{2} \\
            1 \text{ if }V_{1} \cong V_{2}
        \end{cases}\]
        
    \end{thm}
    Let $ \Ch(G) $ or $ \chi G $ denote the ring given by characteristic of representation of $ G $.  \begin{align*}
        RG & \xrightarrow{\chi} \chi G \\
        [V_{\pi}] & \mapsto \chi_{\pi} 
    \end{align*} 
    is a ring homomorphism. 

    \begin{thm}
        $ RG \cong \chi(G) $
    \end{thm}
    \begin{proof}
        We need only show that $ \chi  $ is a monomorphism. Suppose 
        \[ a = \sum_{}^{} a_{i}[V_{i}] \]
        where $ V_{i} $ are irreducible such that $ \chi(a) = 0 $. So 
        \[ \sum_{}^{}a_{i}\chi_{V_{i}} = 0  \]
        this implies 
        \[ \sum_{}^{}a_{i} \delta_{ij} =  \sum_{}^{} a_{i} \left< \chi_{V_{i}}, \chi_{V_{j}} \right>  = 0\]
        for all $ j $. Thus, $ a_{j} = 0 $ hence $ a = 0 $. 
    \end{proof}

    Suppose that $ g \sim h $ in $ G $, so $ g = xhx^{-1} $ for some $ x \in G $. Then $ \chi_{\pi}(g) = \chi_{\pi}(h) $, i.e. $ \chi_{\pi} $ is constant on conjugacy classes. 



    Suppose $ T \subset G $ is torus and $ G $ is compact connected. We say that $ T $ is a maximal torus if 
    \[ T \subset T' \]
    and $ T' $ a torus implies $ T' = T $. 

    \begin{lemma}
        Any $ g \in G $ is contained in a maximal torus.    
    \end{lemma}
    \begin{thm}
        Fix any maximal torus $ T \subset G $. Then 
        \[ G = \bigcup_{x \in G } xTx^{-1}. \]
        
    \end{thm}
    % https://q.uiver.app/?q=WzAsNCxbMCwwLCJSRyJdLFsyLDAsIlJUIl0sWzAsMiwiXFxjaGkoRykiXSxbMiwyLCJcXGNoaShUKSJdLFswLDIsIiIsMCx7InN0eWxlIjp7ImJvZHkiOnsibmFtZSI6InNxdWlnZ2x5In19fV0sWzAsMSwicmVzIiwyXSxbMSwzLCIiLDIseyJzdHlsZSI6eyJib2R5Ijp7Im5hbWUiOiJzcXVpZ2dseSJ9fX1dLFsyLDMsInJlcyJdXQ==
\[\begin{tikzcd}
	RG && RT \\
	\\
	{\chi(G)} && {\chi(T)}
	\arrow[squiggly, from=1-1, to=3-1]
	\arrow["res"', from=1-1, to=1-3]
	\arrow[squiggly, from=1-3, to=3-3]
	\arrow["res", from=3-1, to=3-3]
\end{tikzcd}\]
 where zigzag lines denote isomorphism. Further 
 \[ R(G \times H ) = RG \otimes RH \]
 \[ R(T^{n}) = \Z [\chi_{1},\chi_{1}^{-1}, \dots , \chi_{n},\chi_{n}^{-1}] \]
 
 \section{10 April}
 
 \begin{lemma}
    Suppose that $ \left< \cdot, \cdot \right> $ is a $ G $-invariant Hermitian inner product on $ V_{1} $ where $ G $ is compact. Let $ v_{i} \in V_{i} $. Then we obtain a linear transformation $ T : V_{1} \to V_{2} $ defined by
    \[ T(\omega) = \int_{G} \left< \pi_{1}(g)w,v_{1}  \right> \pi_{2}(g^{-1})v_{2} d g  \in V_{2}\]
    where  $ dg $ is a Haar measure (unimodular here because $ G $ is compact). Then $ T $ is a $ G $-equivariant, i.e. $ T(\pi_{1}(h) \omega) = \pi_{2}(h)T(\omega) $. 
    
 \end{lemma}

 \begin{proof}
    \begin{align*}
        T(\pi_{1}(h)\omega) & = \int_{G} \left< \pi_{1}(g)\pi_{1}(h) \omega, v_{1}  \right> \pi_{2}(g^{-1})(v_{2}) dg \\
        & = \int_{G} \left< \pi_{1}(gh)\omega, v_{1} \right>\pi_{2}(g^{-1})v_{2} dg 
    \end{align*}
    Put $ gh=x $, then $ g = x h^{-1} = \rho_{h}(x) $ and $ dg = dx $. So \begin{align*}
        T(\pi_{1}(h)\omega) & = \int_{G}\left< \pi_{1}(x)\omega, v_{1}  \right> \pi_{2}(h)\pi_{2}(x^{-1})v_{2}dx \\
        & = \pi_{2}(h) \int_{G}\left< \pi_{1}(x) \omega, v_{1} \right> \pi_{2}(x^{-1})v_{2} dx \\
        & = \pi_{2}(h)T(\omega)
    \end{align*}
 \end{proof}
Recall
 \begin{lemma}[Schur's ortho]
    Suppose that $ (\pi_{1},V_{1}) $ and $ (\pi_{2},V_{2}) $ are irreducible. Then every matrix coefficient $ \psi_{\pi_{1},u,v} $ is orthogonal to $ \psi_{\pi_{2},u',v'} $ or $ (\pi_{1},V_{1}) $ is isomorphic to $ (\pi_{2},V_{2}) $. 
 \end{lemma}

 Now 
 \begin{proof}{continuing}
    Assume $ \psi_{\pi_{1},u,v} $ and $ \psi_{\pi_{2},u',v'} $ are not orthogonal. So \begin{align*}
        0  & \neq \int_{G}\left< \pi_{1}(g)u,v \right> \overline{\left< \pi_{2}u',v' \right>}dg \\
        & = \int_{G} \left< \pi_{1}(g)u,v \right> \left< v',\pi_{2}(g)u',v' \right>dg \\
        & = \int_{G}\left< \pi_{1}(g)u,v \right> \left< \pi_{2}(g^{-1})v',u' \right>dg 
    \end{align*}
    which is $ \left< T(u),u' \right> $ hence $ T $ is non-zero so by $ T $ is an isomorphism by Schur's lemma. 
 \end{proof}

 Let $ T $ be a subgroup of $ G $, then we know that there is a map 
 \[ RG \xrightarrow[]{Res_{}^{}}RT \]

 Basic fact : If $ (\pi,V) $ is an irreducible representation of a torus $ T $. Then $ V $ is one-dimensional. 

 \begin{proof}
    Let $ t \in T $. Consider $ \pi(t) : V \to V $. Because $ T $ is abelian, $ \pi(t) $ is  $ T $-linear, i.e. 
    \[ \pi(ts)(v) = \pi(t)(\pi(s)v) = \pi(s)\pi(t)v = \pi(st)(v) \]
    hence by Schur's lemma 
    \[ \pi(t) v = \chi(t)v \]
    for all $ v $ where $ \chi : T \to C^{\times} $ so 
    \[ \chi(st) = \chi(s)\chi(t) \]
    holds. Now 
    \begin{align*}
        \chi(st)v & = \pi(st)v = \pi(s)\pi(t)v \\
        & = \pi(s)(\chi(t)v) = \chi(t)\pi(s)v \\
        & = \chi(t)\chi(s)v
    \end{align*}
    Since every non-zero subspace of $ V $ is a $ T $-representation ( as $ \pi(t) = \chi(t)I $) we must have $ \dim V = 1 $ as $ V $ is irreducible. 
    
 \end{proof}

 \begin{example}
    Let $ G = SU(2) $ with  torus 
    \[ T = \biggl\{\begin{bmatrix}
        e^{i \theta} &  \\
        & e^{-i \theta} \end{bmatrix} : 0 \le \theta \le 2 \pi\biggr\} \]
        where $ T $ is maximal since the only matrices in $ SU(2) $ which commute with every $ \begin{bmatrix}
            e^{i \theta} & \\
            & e^{-i \theta}
        \end{bmatrix} $ is itself diagonal and hence in $ T $. 
 \end{example}
Std : $ V = \C^{2} = V_{1} \oplus V_{2} $ be irreducible where $ V_{i} = \C e_{i} $ and 
\[ \begin{pmatrix}
    e^{i \theta} &  \\
    & e^{-i \theta}
\end{pmatrix}e_{1} = \begin{pmatrix}
    e^{i \theta} & \\
    & e^{-i \theta}
\end{pmatrix} \begin{pmatrix}
    1 \\
    0
\end{pmatrix} = e^{i \theta} \begin{pmatrix}
    1 \\
    0
\end{pmatrix} \]
thus 
\[ \chi_{1} \begin{pmatrix}
    e^{i \theta} & \\
    & e^{-i \theta}
\end{pmatrix} = e^{i \theta}\]
similarly 
\[ \chi_{2}\begin{pmatrix}
    e^{ i \theta} & \\
    & e^{-i \theta}
\end{pmatrix}  = e^{-i \theta}\]


Let $ S^{k}(V  ) $ be the $ k $-th symmetric power of $ V $ which is same as polynomials of degree $ k $ in $ e_{1},e_{2} $. The characters of $ S^{k} $ are 
\[ \begin{pmatrix}
    e^{i \theta} & \\
    & e^{-i \theta}\end{pmatrix}e_{1}^{j}e_{2}^{k-j} = e^{ij \theta}e^{-i(k-j) \theta}e_{1}^{j} e_{2}^{k-j} = e^{i(2j-k) \theta}e_{1}^{j}e_{2}^{k-j}
 \]

\begin{align*}
    RSU(2) & \xrightarrow{Res_{}^{}} RT \\
    S^{k} & \mapsto V_{k} \oplus V_{k-2} \oplus \dots \oplus V_{-k}
\end{align*}
where $ V_{i} \leftrightarrow x_{j} $

\begin{thm}
    The $ S^{k} $ are the only irreducible representations of $ SU(2) $. 
\end{thm}
Known as $ \operatorname{SL}(2) $ theory. 

Let $ G $ be a compact connected Lie group. Let $ T $ be a maximal torus. Then we define the Weyl group $ W = W(G,T) $ of $ G  $ with respect to $ T $ as 
\[ W = N_{G}(T) / T \]
where $ N_{G}(T) = \{g \in G : gTg^{-1} = T\} $
 and $ N_{G} \subset \operatorname{Aut}(T) $ via conjugation. Hence, $ W  $ acts in $ T $ via automorphism. \begin{thm}
    $ W $ is a finite group.
 \end{thm}
 \begin{proof}
    $ W $ acts on $ \operatorname{Lie}(T) $ as linear map. Consider the projection map 
   \begin{align*}
    \R^{n} \cong & \operatorname{Lie}(T) \xrightarrow{p} T \cong \R^{n} / \Z^{n} \\
    (t_{1}, \dots, t_{n}) & \mapsto (e^{2 \pi i t_{1}}, \dots, e^{2 \pi i t_{n}})
   \end{align*}

   % https://q.uiver.app/?q=WzAsNCxbMCwwLCJMaWUoVCkiXSxbMiwwLCJMaWUoVCkiXSxbMCwyLCJUIl0sWzIsMiwiVCJdLFswLDEsIlxcb21lZ2EiXSxbMCwyLCJwIiwyXSxbMiwzLCJcXG9tZWdhIiwyXSxbMSwzLCJwIl1d
\[\begin{tikzcd}
	{Lie(T)} && {Lie(T)} \\
	\\
	T && T
	\arrow["w", from=1-1, to=1-3]
	\arrow["p"', from=1-1, to=3-1]
	\arrow["w"', from=3-1, to=3-3]
	\arrow["p", from=1-3, to=3-3]
\end{tikzcd}\]
    
where $ w(\Z^{n}) = \Z^{n}  $ for all $ w \in W $. Now $ N_{G}(T)    $ is closed in $ G $ and hence compact. So $ W  $ is compact and $ W $ is finite since $ W \subset \operatorname{GL}(n,\Z)$ which is discrete.  
 \end{proof}

 \begin{example}
    Take $ G = U(n) $ and $ T = \biggl\{ \begin{pmatrix} t_{1}&  \hdots &   \\
        & \ddots & \\
    & \hdots & t_{n} \end{pmatrix} : t_{i} \in S^{1} \biggr\} $ Noe that $ U(n) = \bigcup_{g \in U(n)}gT^{-1}g^{-1} $ since given any $ x \in U(n) $, there exists a unitary basis $ \mathcal{U} = u_{1}, \dots, u_{n} $ of $ \C^{n} $ such that the matrix of $ x $ with respect to $ \mathcal{U} $ is diagonal. Take $ g $ to be such that $ g(e_{i}) = u_{i} $ for all $ i $. Then $ (g^{-1}xg)(e_{i}) = g^{-1}x(u_{i}) = \lambda_{i}g^{-1}u_{i}  = \lambda_{i}e_{i}$ 

    On the other hand if $ gTg^{-1}=T $, then choose 
    \[ g \begin{pmatrix}
        \lambda_{1} & & \\
        & \ddots & \\
        & & \lambda_{n}
    \end{pmatrix}g^{-1} \in T \]
    where $ \lambda_{1}, \dots, \lambda_{n} $ are pairwise distinct. This implies $ ge_{i} = z_{i}e_{\sigma(i)} $ for some $ j $,for some $ \sigma \in S_{n} $. Thus $ N(T) $ is a monomial matrix which implies $ N(T) /T \cong S_{n} $. 
 \end{example}
\end{document}
