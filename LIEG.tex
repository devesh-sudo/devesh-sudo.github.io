\PassOptionsToPackage{usenames,svgnames,dvipsnames,table}{xcolor}
\documentclass[12pt,a4paper]{article}

\usepackage[english]{babel}
% Named colors
\usepackage[dvipsnames, table, xcdraw]{xcolor}

% Math stuff
\usepackage{amsmath, amsthm,  amsfonts, amssymb}
\usepackage{tikz, tikz-cd, graphicx}
\usepackage{mathrsfs, mathtools, bm}
\usepackage{halloweenmath}
% Refer
\usepackage[colorlinks,citecolor=blue,urlcolor=blue,bookmarks=false,hypertexnames=true]{hyperref} 
\usepackage[capitalize]{cleveref}

% Change numbering in enumerate
\usepackage[shortlabels]{enumitem}
\crefname{enumi}{part}{parts}
\crefname{figure}{Figure}{Figures}


% pagesetup
\usepackage{graphicx}
\usepackage{fancyhdr}
\pagestyle{fancy}
\setlength{\headheight}{0.75in}
\setlength{\oddsidemargin}{0in}
\setlength{\evensidemargin}{0in}
\setlength{\voffset}{-1.0in}
\setlength{\headsep}{10pt}
\setlength{\textwidth}{6.5in}
\setlength{\headwidth}{6.5in}
\setlength{\textheight}{8.75in}
\setlength{\parskip}{1ex plus 0.5ex minus 0.2ex}
\setlength{\footskip}{0.3in}
\fancyhead[L]{\textbf{Devesh Rajpal}}
\fancyhead[R]{\nouppercase\leftmark}
\usepackage{multicol, titletoc, bookmark, parskip}


% Colored boxes
\usepackage{thmtools}
\usepackage[framemethod=TikZ]{mdframed}
\mdfsetup{skipabove=1em,skipbelow=0em}

\theoremstyle{definition}
\declaretheoremstyle[
    headfont=\bfseries\sffamily\color{ForestGreen!70!black}, bodyfont=\normalfont,
    mdframed={
        linewidth=2pt,
        rightline=false, topline=false, bottomline=false,
        linecolor=ForestGreen, backgroundcolor=ForestGreen!5,
    }
]{greenbox}

\declaretheoremstyle[
    headfont=\bfseries\sffamily\color{NavyBlue!70!black}, bodyfont=\normalfont,
    mdframed={
        linewidth=2pt,
        rightline=false, topline=false, bottomline=false,
        linecolor=NavyBlue, backgroundcolor=NavyBlue!5,
    }
]{bluebox}

\declaretheoremstyle[
    headfont=\bfseries\sffamily\color{Mulberry!70!black}, bodyfont=\normalfont,
    mdframed={
        linewidth=2pt,
        rightline=false, topline=false, bottomline=false,
        linecolor=Mulberry, backgroundcolor=Mulberry!5,
    }
]{purplebox}

\declaretheoremstyle[
    headfont=\bfseries\sffamily\color{RawSienna!70!black}, bodyfont=\normalfont,
    mdframed={
        linewidth=2pt,
        rightline=false, topline=false, bottomline=false,
        linecolor=RawSienna, backgroundcolor=RawSienna!5,
    }
]{redbox}

\declaretheoremstyle[
    headfont=\bfseries\sffamily\color{CadetBlue!70!black}, bodyfont=\normalfont,
    numbered=no,
    mdframed={
        linewidth=2pt,
        rightline=false, topline=false, bottomline=false,
        linecolor=CadetBlue, backgroundcolor=CadetBlue!4,
    },
    qed=\qedsymbol
]{proofbox}
\declaretheoremstyle[
    headfont=\bfseries\sffamily\color{CadetBlue!70!black}, bodyfont=\normalfont,
    mdframed={
        linewidth=2pt,
        rightline=false, topline=false, bottomline=false,
        linecolor=CadetBlue, backgroundcolor=CadetBlue!10,
    }
]{CadetBluebox}


\renewenvironment{proof}[1][\proofname]{\vspace{-15pt}\begin{myproof}}{\end{myproof}}
\declaretheorem[style = CadetBluebox, name = Theorem, numberwithin=section]{thm}
\declaretheorem[style = proofbox, name=Proof, numbered=no]{myproof}
\declaretheorem[style = greenbox, name = Definition, numberwithin=section]{defn}
\declaretheorem[style = greenbox, name = Notation]{notation}
\declaretheorem[style = redbox, name = Proposition,sibling=thm]{proposition}

\declaretheorem[style = CadetBluebox, name = Lemma, sibling=thm]{lemma}
\declaretheorem[style = redbox, name = Corollary, numbered = no]{corollary}
\declaretheorem[style = bluebox, name = Observation, numbered = no]{observation}
\declaretheorem[style = bluebox, name = Example, numbered = no]{example}
\declaretheorem[style = bluebox, name = Remark, numbered = no]{remark}
\declaretheorem[style = purplebox, name = Problem, numbered = no]{problem}
\declaretheorem[style = purplebox, name = Exercise, numbered = no]{exercise}
\theoremstyle{greenbox}
\newtheorem*{motivation}{Motivation}
\newtheorem*{recall}{Recall}
\newtheorem*{note}{Note}


% Figure support.
\usepackage{pdfpages}
\usepackage{transparent}
\graphicspath{{figures/}{images/}} %typical directories

%math
\newcommand{\R}{\mathbb{R}}
\newcommand{\Z}{\mathbb{Z}}
\newcommand{\Q}{\mathbb{Q}}
\newcommand{\C}{\mathbb{C}}
\newcommand{\Ho}{\mathbb{H}}
\newcommand{\dou}{\partial}
\newcommand{\half}{\ensuremath{\frac{1}{2}}}
\newcommand{\pt}{\partial_t}
\newcommand{\dt}{\frac{\partial}{\partial_t}}
\newcommand{\define}{\doteqdot}


\title{\LARGE Lie Groups}
\author{\large Devesh Rajpal}
\date{}
\pdfsuppresswarningpagegroup=1
\begin{document}
    \maketitle
    \tableofcontents
    \section{4th January 23}
    One can study Lie Groups from several points of view. The course is aimed to understand the structure of Lie Groups. 
    \begin{defn}
        A smooth manifold $ M $ is a Hausdorff space which is locally Euclidean with a smooth atlas i.e. (i) given any $ x \in M $, $ \exists  $ a chart $ (U, \phi) , \, x \in U \subset M$ with  $ \phi : U \to \phi (U) $ open in $ \R^{m} $.


        (ii) We have collection $ \{(U, \phi)\} $ of charts such that 
        \[ \phi ( U \cap V) \xrightarrow{\psi \circ \phi^{-1}} \psi ( U \cap V)\]
        is  a diffeomorphism.
        
    \end{defn}

    Suppose $ f : M \to N $ is a continuous map between manifolds. We say that $  f  $ is smooth if for $ (U, \phi) \in \mathcal{q}(M) $, $ (V, \psi) \in \mathcal{q}(N) $ such that $ f(U) \subset V $ and $ \psi \circ f \circ  \phi^{-1} $ is smooth. 
    
    TO DO : Construction of tangent bundle and vector bundle

    \section{9th Jan 2023}

    \begin{defn}
        $ G $ is a Lie group if \begin{enumerate}
            \item $ G $ is a smooth manifold
            \item $ G $ is also a group s.t 
            \begin{align*}
                \mu : G \times G \to G \\
                ( g,h) \mapsto gh
            \end{align*} and 
            \begin{align*}
                i : G \to G \\
                g \mapsto g^{-1}
            \end{align*}
            are smooth maps.
        \end{enumerate}
    \end{defn}
    \begin{defn}
        A real (or complex) vector space $ V $ together with a bilinear map \begin{align*}
            [,] : V \times V \to V
        \end{align*}
        is called a \textbf{Lie Algebra} if \begin{enumerate}
            \item $[X,Y] = -[Y,X]$ - skew symmetry
            \item $[[X,Y],Z]+ [[Y,Z],X]+ [[Z,X],Y] = 0 $ - Jacobi identity
        \end{enumerate}
    \end{defn}

    \begin{example}
        \begin{enumerate}
            \item $ \left( \R, + \right) , (\C, +)$, $ V $ any f.d vector space over $ \R $ or $ \C $. 
            \item $ (\R^{\times}, \cdot), (\C^{\times}, \cdot) $
            \item $ S^{1} = \{ z \in \C^{\times}| |z| =1\} $
            \item $ \operatorname{GL}_{n}(\R) $, $ \operatorname{GL}_{n}(\C) $
            \item $ \R^{n} / \Z^{n} \cong \left( \R^{n} /\Z^{n} \right) \cong (S^{1})^{n}$
            \item Suppose $ \Gamma \subset V $ is a discrete subgroup. Then $ V / \Gamma  $ is a Lie group.
            \item $ N = $ unipotent upper triangular matrices, $ B =  $ upper triangular matrices. As manifolds $ N \cong \R^{\binom{n}{2}} $ and $ B \cong (\R^{\times})^{n} \times N $.
            \item $ \operatorname{SL}_{n}(\R) = \{X \in \operatorname{GL}_{n}(\R) | \det X = 1\} $, $ \operatorname{SL}_{n}(\C) $. 
            \item $ O(n) $, $ SO(n) $.
            \item $ U(n) $, $ SU(n) $.
            \item $ \Ho^{\times} $, $ S^{3}  $ with quaternion multiplication.
            \item $ Sp(n) = \{X \in \operatorname{GL}_{n}(\R)| X \text{ preserves quaternion structure as a subset of } \operatorname{Aut}_{\Ho}\Ho^{n}\} $
        \end{enumerate}
    \end{example}
    \begin{problem}
        $ V / \Gamma  \cong \R^{k} \times  (S^{1})^{n-k}$ for $ n $-dimensional vector space $  V$. 
    \end{problem}
    \begin{thm}
        Suppose $ G $ is a compact, connected, simple Lie group. Then $ G $ is locally isomorphic to \begin{enumerate}
            \item $ SU(n) , n \ge 2$ denoted by $ A_{n-1} $
            \item $ SO(2n+1), n \ge 2 $ denoted by $ B_{n} $
            \item $ Sp(n), n \ge 1 $ denoted by $ C_{n} $
            \item $ SO(2n), n\ge 2 $ denoted by $ D_{n} $
        \end{enumerate}
        or one of the following exceptional Lie group $ G_{2}, F_{4}, E_{6}, E_{7}, E_{8} $.

    \end{thm}
    \begin{problem}
        Prove that $ \operatorname{SL}_{n}(\R) $ and $ O(n) $ are smooth manifold, hence Lie groups.
    \end{problem}

    Examples of Lie algebra - \begin{example}
       \begin{enumerate}
        \item $ (V, [\cdot, \cdot] \equiv 0 )$ is called trivial Lie algebra. 
        \item $ (\mathfrak{gl}_{n}(\R), [A,B ]  = AB - BA )$, $ \mathfrak{gl}_{n}(\C)$
        \item $ \mathfrak{sl}_n(\R) $ ($ \mathfrak{sl}_n (\C) $) is the Lie subalgebra of $ \mathfrak{gl}_{n}(\R) $ ($ \mathfrak{gl}_{n}(\C) $)  consisting of trace $ 0 $.
        \item $ \mathfrak{so}_n $ is Lie subalgebra of $ \mathfrak{gl}_{n}(\R) $ consisting of skew-symmetric matrices.
       \end{enumerate}
    \end{example}
    \begin{defn}
        A vector field $ X $ on a Lie group $ G $ is called left invariant if $ (L_{g})_{*}(X_{h}) = X_{gh} $
    \end{defn}

    \section{11th Jan 2023}

    Recall $ \Ho =  \{a+bi+cj+dk : (a,b,c,d) \in \R^{4},  \, i^{2} = -1, j^{2} = -1, k^{2} = -1,  ij=k, jk = l, ki=j \} $ is the quaternion division algebra  with the norm 
    \[ ||a+ bi+ cj+dk||^{2} = a^{2}+ b^{2}+c^{2}+d^{2} \] which satisfies $ ||q_{1} \cdot  q_{2}|| = ||q_{1}|| \cdot ||q_{2}||$

    We can put this multiplication on $ S^{3} \cong SU(2) $ to get a compact Lie group. To get the isomorphism $ SU(2) \cong S^{3} $, we define a map 
    \begin{align*}
        \phi :S^{3} & \to SU(2)\\
            (a,b,c,d) & \mapsto \begin{bmatrix}
                a+bi & c+di\\
                -(c-di)&  a-bi
            \end{bmatrix}
    \end{align*}
    which is an algebra isomorphism. 

    \begin{defn}
        The Lie algebra of $ G $ is the space of all left-invariant vector fields on $ G $. 
        
    \end{defn}
    We have an  isomorphism \begin{align*}
        \mathfrak{g}  = \operatorname{Lie}(G) &\to T_{e}G\\ 
        X &\mapsto X_{e}
    \end{align*}
    \begin{example}
        Let $ G = \R^{n} $, with identity element $ 0 \in \R^{n} $ and left-invariant vector fields $ \{ \frac{ \partial}{ \partial x_{1}}, \ldots, \frac{ \partial}{ \partial x_{n}}\} $. Then the Lie bracket is \begin{align*}
            [\cdot, \cdot ] \equiv 0
        \end{align*}
    \end{example}
    \begin{remark}
        In general for any abelian Lie group $ G $, the Lie bracket is $ [\cdot, \cdot ] \equiv 0 $.
    \end{remark}
    \begin{thm}
        Let $ G $ be a connected Lie group. Then \begin{enumerate}
            \item Lie$ (G)  = \mathfrak{g}$ is isomorphic as a vector space to $ T_{e}(G) $.
            \item Left-invariant vector fields are smooth. 
            \item Lie$ (G) $ is closed under Lie bracket. 
        \end{enumerate}
    \end{thm}
    \begin{proof} 1. 
        Let $ X $ be a left-invariant vector field on $ G $. We need to show that $ Xf $ is smooth for each $ f \in C^{\infty}(G) $. 
        \begin{align*}
            (Xf)(g) & = X_{g}f\\
             & = (d \lambda_{g} X_{e})f \\
             & = X_{e}( f \circ \lambda_{g})
        \end{align*}
        To show that $ Xf $ is smooth, it suffices to show that $ X_{e}(f \circ \lambda_{g}) $ is smooth. We realize $ X_{e}(f \circ \lambda_{g}) $ as evaluation of a smooth function on a smooth function. 

        Let $ Y $ be a smooth vector field on $ G $ such that $ Y_{e} = X_{e}$ \begin{align*}
            Y_{e}(f \circ \lambda_{g}) = X_{e}( f \circ \lambda_{g})
        \end{align*}
        We look at $ \lambda_{g} $ as the composition of \begin{align*}
            G &\xrightarrow{i_{g}^{2}} G \times G 
            \xrightarrow{\mu} G \\
            x & \mapsto (g,x)\mapsto gx
        \end{align*}
        Regard $ Y $ as the vector field $ (0,Y) $ on $ G \times G $. Now \begin{align*}
            (0,Y)(f \circ \mu) \circ i_{e}^{1}(g) & = (0,Y)_{(g,e)} ( f\circ \mu)\\
            & =0_{g}( f\circ \mu \circ i_{g}^{1})+ Y_{e}( f\circ \mu \circ i_{g}^{2}) \\
            & = Y_{e}( f \circ \lambda_{g})
        \end{align*}
        which proves the smoothness.

        2. Let $ X,Y  $ left-invariant vector fields on $ G $. We must show that $ [X,Y] $  is a left-invariant vector field. \begin{align*}
            d \lambda_{g}([X,Y]_{e})f & = [X,Y]_{g}f \\
            & = [X,Y]_{e}( f\circ \lambda_{g}) \\
            & = X_{e}(Y(f\circ \lambda_{g})) - Y_{e}(X(f \circ \lambda_{g})) \\
            & = X_{e}(d \lambda_{g}(Yf)) - Y_{e}(d \lambda_{g}(Yf)) 
        \end{align*}
    \end{proof}
    
\end{document}