\PassOptionsToPackage{usenames,svgnames,dvipsnames,table}{xcolor}
\documentclass[12pt,a4paper]{article}



\usepackage{epigraph}
\usepackage{comment}
\usepackage[english]{babel}
% Named colors
\usepackage[dvipsnames, table, xcdraw]{xcolor}

% Math stuff
\usepackage{amsmath, amsthm,  amsfonts, amssymb}
\usepackage{tikz, tikz-cd, graphicx}
\usepackage{mathrsfs, mathtools, bm}
\usepackage{halloweenmath}
\usepackage{enumerate}
% Refer
\usepackage[colorlinks,citecolor=blue,urlcolor=blue,bookmarks=false,hypertexnames=true]{hyperref} 
\usepackage[capitalize]{cleveref}

% Change numbering in enumerate
\usepackage[shortlabels]{enumitem}
\crefname{enumi}{part}{parts}
\crefname{figure}{Figure}{Figures}


% pagesetup
\usepackage{graphicx}
\usepackage{fancyhdr}
\pagestyle{fancy}
\setlength{\headheight}{0.75in}
\setlength{\oddsidemargin}{0in}
\setlength{\evensidemargin}{0in}
\setlength{\voffset}{-1.0in}
\setlength{\headsep}{10pt}
\setlength{\textwidth}{6.5in}
\setlength{\headwidth}{6.5in}
\setlength{\textheight}{8.75in}
\setlength{\parskip}{1ex plus 0.5ex minus 0.2ex}
\setlength{\footskip}{0.3in}
\fancyhead[L]{\textbf{Devesh Rajpal}}
\fancyhead[R]{\nouppercase\leftmark}
\usepackage{multicol, titletoc, bookmark, parskip}


% Colored boxes
\usepackage{thmtools}
\usepackage[framemethod=TikZ]{mdframed}
\mdfsetup{skipabove=1em,skipbelow=0em}

\theoremstyle{definition}
\declaretheoremstyle[
    headfont=\bfseries\sffamily\color{ForestGreen!70!black}, bodyfont=\normalfont,
    mdframed={
        linewidth=2pt,
        rightline=false, topline=false, bottomline=false,
        linecolor=ForestGreen, backgroundcolor=ForestGreen!5,
    }
]{greenbox}

\declaretheoremstyle[
    headfont=\bfseries\sffamily\color{NavyBlue!70!black}, bodyfont=\normalfont,
    mdframed={
        linewidth=2pt,
        rightline=false, topline=false, bottomline=false,
        linecolor=NavyBlue, backgroundcolor=NavyBlue!5,
    }
]{bluebox}

\declaretheoremstyle[
    headfont=\bfseries\sffamily\color{Mulberry!70!black}, bodyfont=\normalfont,
    mdframed={
        linewidth=2pt,
        rightline=false, topline=false, bottomline=false,
        linecolor=Mulberry, backgroundcolor=Mulberry!5,
    }
]{purplebox}

\declaretheoremstyle[
    headfont=\bfseries\sffamily\color{RawSienna!70!black}, bodyfont=\normalfont,
    mdframed={
        linewidth=2pt,
        rightline=false, topline=false, bottomline=false,
        linecolor=RawSienna, backgroundcolor=RawSienna!5,
    }
]{redbox}

\declaretheoremstyle[
    headfont=\bfseries\sffamily\color{CadetBlue!70!black}, bodyfont=\normalfont,
    numbered=no,
    mdframed={
        linewidth=2pt,
        rightline=false, topline=false, bottomline=false,
        linecolor=CadetBlue, backgroundcolor=CadetBlue!4,
    },
    qed=\qedsymbol
]{proofbox}
\declaretheoremstyle[
    headfont=\bfseries\sffamily\color{CadetBlue!70!black}, bodyfont =\normalfont, 
    mdframed={
        linewidth=2pt,
        rightline=false, topline=false, bottomline=false,
        linecolor=CadetBlue, backgroundcolor=CadetBlue!10,
    }
]{CadetBluebox}


\renewenvironment{proof}[1][\proofname]{\vspace{-15pt}\begin{myproof}}{\end{myproof}}
\declaretheorem[style = CadetBluebox, name = Theorem, numberwithin=section]{thm}
\declaretheorem[style = proofbox, name=Proof, numbered=no]{myproof}
\declaretheorem[style = greenbox, name = Definition, numberwithin=section]{defn}
\declaretheorem[style = greenbox, name = Notation]{notation}
\declaretheorem[style = redbox, name = Proposition,sibling=thm]{proposition}

\declaretheorem[style = CadetBluebox, name = Lemma, sibling=thm]{lemma}
\declaretheorem[style = redbox, name = Corollary, numbered = no]{corollary}
\declaretheorem[style = bluebox, name = Observation, numbered = no]{observation}
\declaretheorem[style = bluebox, name = Example, numbered = no]{example}
\declaretheorem[style = bluebox, name = Remark, numbered = no]{remark}
\declaretheorem[style = purplebox, name = Problem, numbered = no]{problem}
\declaretheorem[style = purplebox, name = Exercise, numbered = no]{exercise}
\theoremstyle{greenbox}
\newtheorem*{motivation}{Motivation}
\newtheorem*{recall}{Recall}
\newtheorem*{note}{Note}


% Figure support.
\usepackage{pdfpages}
\usepackage{transparent}
\graphicspath{{figures/}{images/}} %typical directories

%math
\newcommand{\R}{\mathbb{R}}
\newcommand{\Z}{\mathbb{Z}}
\newcommand{\Q}{\mathbb{Q}}
\newcommand{\dou}{\partial}
\newcommand{\half}{\ensuremath{\frac{1}{2}}}
\newcommand{\pt}{\partial_t}
\newcommand{\dt}{\frac{\partial}{\partial_t}}
\newcommand{\define}{\doteqdot}
\newcommand{\dohi}{\frac{\partial}{\partial x^i}}
\newcommand{\dohj}{\frac{\partial}{\partial x^j}}
\newcommand{\dohk}{\frac{\partial}{\partial x^k}}
\newcommand{\dohl}{\frac{\partial}{\partial x^l}}
\newcommand{\doha}{\frac{\partial}{\partial x^a}}
\newcommand{\dohb}{\frac{\partial}{\partial x^b}}
\newcommand{\nj}{\text{inj}}
\newcommand{\Rn}{\mathbb{R}^{n+1}}


\title{\LARGE Differential Geometry}
\author{\large Devesh Rajpal}
\date{}
\pdfsuppresswarningpagegroup=1
\begin{document}
    \maketitle
    \tableofcontents
    \section{Oct 13, 2022}
    \subsection*{Levi-Civita connection}
    \bigskip
    \begin{thm}
        Let $ (M,g) $  be a Riemannian manifold. There exists a unique connection $ \nabla $ on $ M $ which satisfies the following for any vector fields $ X,Y,Z \in \Gamma(TM)$ 
        \begin{enumerate}
            \item Compatibility with metric : \[ Xg(Y,Z)  = g(\nabla_{X}Y, Z)+ g(Y, \nabla_{X}Z)\]
            \item Torsion-free : \[ \nabla_{X}Y - \nabla_{Y}X = [X,Y] \]
        \end{enumerate}
        Such a connection is called the \textbf{Levi-Civita} connection on $ M $.
    \end{thm}
    \begin{proof}
        Assume such a connection exists, then 
    \begin{equation}
        Xg(Y,Z) = g(\nabla_{X}Y,Z)+ g(Y,\nabla_{X}Z) \label{eq1}
    \end{equation}
    \begin{equation}
        Yg(Z,X) = g(\nabla_{Y}Z,X)+ g(Z,\nabla_{Y}X) \label{eq2}
    \end{equation}
    \begin{equation}
        Zg(X,Y) = g(\nabla_{Z}X,Y)+ g(X,\nabla_{Z}Y) \label{eq3}
    \end{equation}
    Adding \cref{eq1} and \cref{eq2} and subtracting \cref{eq3} from the sum 
    \[ Xg(Y,Z)+ Yg(Z,Z)- Zg(X,Y) = g(Y,[X,Z])+ g(X,[Y,Z]) + 2g(\nabla_{X}Y,Z) - g(Z,[X,Y])\]
    Or 
    \begin{equation}
        2g(\nabla_{X}Y,Z) = Xg(Y,Z)+ Yg(Z,Z)- Zg(X,Y) + g(Z,[X,Y]) - g(Y,[X,Z])- g(X,[Y,Z])
    \end{equation}
    This gives us a formula for determining the connection $ \nabla $ which proves the existence and uniqueness. Further it can be proved that properties 1. and 2. are satisfied with this formula.
    \end{proof}


    Let $ \gamma : (\R, \delta) \to (M,g) $ be a smooth curve on $ M $. Let $ X \in \Gamma(TM) $ be a vector field. Define $ X_{\gamma}(t) = X(\gamma(t))$ to be its restriction on $ \gamma $. Then 
    \[ \frac{D}{dt} X(t) \define \nabla_{\gamma'(t)}X|_{\gamma(t)}  \]
    
    which is called covariant differentiation along the curve $ \gamma $. 
    \begin{defn}
        A curve $ \gamma :(a,b) \to M$ is called a \textbf{geodesic}  if 
        \[ \nabla_{\gamma'(t)}\gamma'(t) = 0 \qquad \forall t \in (a,b) \]
    \end{defn}

    \section{Oct 18, 2022}

    \subsection*{Parallel Transport along a curve}  

    Let $ (M,g) $ be a smooth Riemannian manifold. Let $ \gamma : \R \to M $ be a smooth curve and $ X \in \Gamma(TM) $ is a vector field. Restricting $ X $ to $ \gamma $ we get a vector field along $ \gamma $ which we define by $ X(t) \define X|_{\gamma(t)} $. Let 
    \[ \frac{D}{dt} X(t) \define (\nabla_{\gamma'(t)}X)|_{\gamma(t)} \]
    \begin{defn}
        A vector field $ X $ along $ \gamma $ is called \textbf{parallel}  along $ \gamma  $ if $ \frac{D}{dt} X(t)  =0$. If $ t_{0}< t_{1} $, we say that $ X(t_{1}) $ is obtained from $ X(t_{0}) $ by parallel transport along $ \gamma $. 
    \end{defn}
    In a local chart $ \{x^{i}\} $, let $ \gamma  $ be given by $ \gamma(t) = (\gamma^{1}(t), \dots, \gamma^{n}(t)) $ and $ X|_{\gamma(t)} = X^{i} \frac{ \partial}{ \partial x^{i}}\big|_{\gamma(t)} $ or 
    \[ X(t) = X^{i}(t) \frac{ \partial}{ \partial x^{i}}\bigg|_{\gamma(t)}\]

    The equation for parallel transport in local coordinates becomes 
    \begin{align}
        \frac{D}{dt} X(t) & =0 \\
        \left( \nabla_{\gamma_{*}( \frac{ \partial}{ \partial t})}X \right)\bigg|_{\gamma(t)} & = 0 \\
        \nabla_{\gamma'(t)}\left( X^{i} \dohi \right)\bigg|_{\gamma(t)} & =0\\ 
        \frac{dX^{i}}{dt} \dohi\bigg|_{\gamma(t)} + X^{i}\nabla_{\gamma'(t)}\dohi\bigg|_{\gamma(t)} & = 0\\ 
        \frac{dX^{i}}{dt} \dohi\bigg|_{\gamma(t)} +X^{i} \frac{d\gamma^{k}(t)}{dt} \nabla_{\frac{ \partial}{ \partial x^{k}}}\frac{ \partial}{ \partial x^{i}} & = 0\\
        \frac{dX^{i}}{dt} \dohi\bigg|_{\gamma(t)} +X^{i} \frac{d\gamma^{k}(t)}{dt} \Gamma_{ki}^{l} \frac{ \partial}{ \partial x^{l}}\bigg|_{\gamma(t)} &  = 0\\ 
        \left( \frac{dX^{l}}{dt} +X^{i} \frac{d\gamma^{k}(t)}{dt} \Gamma_{ki}^{l}\right)\frac{ \partial}{ \partial x^{l}}\bigg|_{\gamma(t)} & =0 \label{parallel}
    \end{align}
    which is a system of linear equations hence there exits a unique solution for all the time it is defined. 
    
    If $ g = (g_{ij}) $ in the chart $ \{x^{i}\} $, then from the compatibility of $ \nabla $, we get 
    \begin{align*}
        \dohi g_{jk} &= g\left(\nabla_{\dohi} \dohj, \dohk\right)+ g\left(\dohj, \nabla_{\dohi}\dohk\right)\\ 
        & = \Gamma_{ij}^{l}g_{kl}+ \Gamma_{ik}^{l}g_{lj}
    \end{align*}
    Using other cyclic relations, we get 
    \[ \Gamma_{ij}^{k} = g^{kl}\left(\frac{ \partial g_{lj}}{ \partial x^{i}} + \frac{ \partial g_{il}}{ \partial x^{j}} - \frac{ \partial g_{ij}}{ \partial x^{l}}\right) \]
    
    \subsection*{Geometric interpretation of torsion}   
    
    For any connection $ \nabla $, the torsion tensor is defined by 
    \[ \tau(X,Y) = \nabla_{X}Y - \nabla_{Y}X - [X,Y]\]
    
    Let $ \gamma(s,t) $ be variation of a curve. Then 
    \[ \left[ \frac{D}{ds}, \frac{D}{dt}\right] = 0 \Longleftrightarrow \tau \equiv 0 \]
    
    \section{Oct 20, 2022}

    \subsection*{Geodesics and the Exponential map}
    Let $ (M,g) $ be a smooth Riemannian manifold and $ \gamma : [0, \tau] \to M $ be a smooth curve on $ M $. Then 

    \[ \frac{d}{dt}g\left(\gamma'(t), \gamma'(t)\right) = 2g\left( \frac{D}{dt}(\gamma'(t)), \gamma'(t)\right) \]
    If $ \gamma $ is a geodesic, then parallel transport along $ \gamma $ is denoted by $ P_{\gamma(t_{0}) \to \gamma(t_{1})} X(t) $ for $ X \in \Gamma(TM) $. Now we want to find an expression of $ P $ in local coordinates. Let $ X = X^{i}\dohi $ be a vector at $ \gamma(0) $. We want to solve the differential equation 
    \[ \frac{D}{dt}X(t) = 0 \]
    along $ \gamma $ where $ \frac{D}{dt} = \nabla_{\gamma'(t)} $.  This was done in \cref{parallel} which is a linear system of equation and the solution exists for while we are in a coordinate patch. To get a general solution, we cover the image of $ \gamma $ by finitely many coordinate patches and repeat the process (since the image is compact, we can cover it by finitely many patches). 

    Now we use this to define the exponential map at a point. 
    \begin{proposition}[Do Carmo Proposition 2.7]
        Given $ p \in M $, there exists a neighborhood $ V $ of $ p \in M $, a number $ \epsilon >0 $ and a $ C^{\infty} $ mapping $ \gamma :(-2,2) \times U \to M $, $ U = \{(q,w) \in TM |\, q \in V, w \in T_{q}M, |w| < \epsilon\} $ such that $ \gamma(t,p,w) , t \in (-2,2) $ is the unique geodesic of $ M $ which at the instant $ t=0 $, passes through $ q $ with velocity $ w $, for which $ q \in V $ and for every $ w \in T_{q}M $, with $ |w| < \epsilon $. 

    \end{proposition}
    \begin{defn}   
        Let $ p \in M $ and $ V, \epsilon, U $ as defined in the previous proposition, then the \textbf{exponential map} is defined by 
        \[  \exp (p,v) = \gamma\left(1,p, \frac{v}{|v|}\right), \quad \text{ for all }(p,v) \in U\]
         
    \end{defn} 
    So exponential map sends a vector $ v $ to a point  which is obtained by going out the length $ |v| $ in the direction of $ v $ by the unique geodesic. Using this we can write exponential map as 
    \[ \exp_{p} : B_{\epsilon}(0) \subset T_{p}M \to M \]
    by $ \exp_{p}(v) = \exp(p,v) $ where $ B_{\epsilon} $ the open ball with center at origin $ 0 $ of $ T_{p}M $ and of radius $ \epsilon $.     
    \begin{defn}   
        The \textbf{injectivity radius} at point $ p $ of the manifold $ (M,g) $ is defined to be the supremum of $ \epsilon $'s for which the exponential map  $\exp_{p} : B_{\epsilon}(0) \to M$ is a diffeomorphism. It is denoted by $ \text{inj}_{M}(p) $.
    \end{defn} 
    Further, define injectivity radius of $ M $ to be the infimum of $\nj_{M}(p) $ among all the points, 
    \[ \nj M \define \inf_{p \in M} \nj_{M}(p) \]
    \begin{example}
        1. The injectivity radius of $ n $-sphere of radius $ R $ is $ \pi R $. \\
        2. The injectivity radius of $ \R^{n} $ is $ \infty $. 
    \end{example}
    \begin{remark}
        There exists a complete Riemannian manifold with $ \nj(M) = 0 $. For sphere of radius $ R $ we know that $ \nj(p) = \pi R $. Attach a sphere of radius $ \frac{1}{2n} $  at each point $ (n,0) $ on the plane $ \R^{2} $ by cutting out a small open set at south pole. Then we get a (geodesic/metric) complete manifold where the infimum of the injectivity radius at the north pole of the attached spheres is $ 0 $. 
    \end{remark}
    \section{November 1, 2022}
    Let $ (M^{n},g) \hookrightarrow (\Rn, \delta) $ be an isometric embedding. Let $ N $ be a local normal field, and define the shape operator by
    \[ S(X) \define - D_{X}N \]
    To define a connection on $ M^{n} $, we project the Levi-Civita connection of $ \Rn $ into the tangent space of the hypersurface. Let $ X_{p} \in T_{p}M^{n} $ and $ Y \in \Gamma(TM^{n}|_{U}) $, define 
    \[ \nabla_{X}Y \define (D_{X}Y)^{T} = D_{X}Y - \left< D_{X}Y,N \right>N \]
    From the relation $ 0 = X \left< Y,N \right> = \left< D_{X}Y,N \right> + \left< Y,D_{X}N \right> $ we can also write 
    \[ \nabla_{X}Y = D_{X}Y - \left< S(X),Y \right>N \]
    
    It is easy to prove that $ \nabla $ is the unique Levi-Civita connection on $ M^{n} $.
    
    Now we define curvature on a Riemannian manifold. Riemann curvature is a tensor defined by 
    \begin{align*}
        R(X,Y)Z & \define \nabla_{X} \nabla_{Y}Z - \nabla_{Y}\nabla_{X}Z - \nabla_{[X,Y]}Z\\
        & = [\nabla_{X},\nabla_{Y}]Z - \nabla_{[X,Y]}Z
    \end{align*}
    for vectors $ X,Y,Z \in T_{p}M^{n} $ (prove that it is independent of the local extension).
    
    For hypersurfaces another formula holds, 
    \[ R(X,Y)Z = \left< S(Y),Z \right>S(X) - \left< S(X),Z \right>S(Y) \]
    which can be remembered mnemonically, 
    \begin{align*}
        \left< R(X,Y)Z,W \right> &= \det \begin{bmatrix}
            \left< S(Y),Z \right> & \left< S(Y),W \right>\\ 
            \left< S(X),Z \right> & \left< S(X),W \right>
        \end{bmatrix} 
    \end{align*}
    The Riemann curvature tensor $ \left< R(X,Y)Z,W \right> $ is a $ (0,4) $ tensor which satisfies the following properties   
    \begin{enumerate}
        
        \item $ \left< R(X,Y)Z,W \right> = - \left< R(Y,X)Z,W \right> $
        \item $ \left< R(X,Y)Z,W \right> = - \left< R(X,Y)W,Z \right> $
        \item $ \left< R(X,Y)Z,W \right> = \left< R(Z,W)X,Y \right> $
    \end{enumerate}

    With the help of first two properties we can also define curvature as a map on the wedge product $ Rm : \wedge^{2}TM^{n} \to \wedge^{2}T^{*}M^{n}  $ by the formula 
    \[ (Rm(X \wedge Y))(Z \wedge W) = \left<  R(X,Y)Z,W\right>  \]
    It is more standard to consider the curvature operator as the map $ Rm : \wedge^{2}T^{*}M^{n} \to \wedge^{2}T^{*}M^{n}$ after lowering/raising appropriate indices. In the case of surfaces, the wedge product is one-dimensional, so the map is determined by a scalar which is equal to the Gaussian curvature/sectional curvature. 

    \section{November 3, 2022}

    Let $ (M^{n},g)  \hookrightarrow (\Rn, \delta)$ be a smooth hypersurface in $ \Rn $. We are going to define the Gauss map on the hypersurface, under the assumption that $ M^{n} $  is compact and without boundary. Given any point $ p \in M^{n} $, let $ N_{p} \in \Rn $ denote \textit{a} (there are two choices) unit normal at $ p $. Define the \textbf{Gauss map}, $ \mathcal{G} : M^{n} \to S^{n} $ by 
    \[ \mathcal{G} : p \mapsto N_{p}  \]
    for a continuous choice of normal which is possible because the hypersurface is oriented. Given a map from oriented, compact manifold to sphere, one has a notion of degree of the map. The definition comes from differential topology and is also equal to the degree of the top homology map. 
    \newpage
    \begin{thm}[Gauss-Bonnet]
        Let $ M^2 $ be an embedded surface in $ \R^{3} $. Then the degree of the Gauss map is equal to 
        \[ \deg( \mathcal{G}) = \frac{1}{2 \pi} \int_{M^{2}}K dV  = \chi(M^{2}) = 2-2g\]
        for a genus $ g $ surface.
    \end{thm}
    \begin{remark}
        This proves that the integral of curvature (metric dependent) is an integer which depends only on the topology of the surface.
    \end{remark}
    Note : Exam may contain simple applications of Gauss-Bonnet.
    Another reference : Problems in Geometry and Topology.

    \section{November 8, 2022}

    Let $ (M^{2},g) $ be a smooth surface in $ \R^{3} $ and $ \gamma : \R \to M $ be a smooth curve. The \textbf{geodesic curvature} denoted by $ \kappa_{g}(s) $ of $ \gamma $ at $ \gamma(s) $ is defined as
    \[ \kappa_{g}(s) = \bigg| \frac{D}{ds}\gamma'(s)\bigg|_{g} \]
     
    which is different from the previous definition 
    \[ \kappa_{g}(t) = \frac{\det \left< \gamma''(t), \gamma'(t), N_{\gamma(t)} \right>}{||\gamma'(t)||^{3}} \]
    which was obtained using Frenet-Serret frame.

    \begin{thm}
        Let $ (M,g)  \subset \R^{3}$ be a compact orientable smooth surface with induced metric $ g $. Denote its connected boundary  by $ \partial M $. Then 
        \[ \int_{M}\kappa dA + \int_{ \partial M} \kappa_{g}ds + \sum_{i=1}^{r} \theta_{i}  = 2 \pi \chi(M)\]
        where $ \theta_{i} $ are the exterior angle in $ \partial M $.
    \end{thm}
    
\end{document}