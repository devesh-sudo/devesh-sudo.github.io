\documentclass[12pt,a4paper]{article}
\usepackage[utf8]{inputenc}
%\usepackage{courier}


%\usepackage{fullpage}
\usepackage{amsmath}
\usepackage{amsthm}
\usepackage{amsfonts}
\usepackage{amssymb}
\usepackage{mathrsfs, mathtools, bm}
\usepackage[colorlinks,citecolor=blue,urlcolor=blue,bookmarks=false,hypertexnames=true]{hyperref} 
\usepackage{graphicx}
\usepackage{xcolor}

\usepackage{graphicx}
\usepackage{fancyhdr}
%\pagestyle{fancy}
\setlength{\headheight}{0.75in}
\setlength{\oddsidemargin}{0in}
\setlength{\evensidemargin}{0in}
\setlength{\voffset}{-1.0in}
\setlength{\headsep}{10pt}
\setlength{\textwidth}{6.5in}
\setlength{\headwidth}{6.5in}
\setlength{\textheight}{8.75in}
\setlength{\parskip}{1ex plus 0.5ex minus 0.2ex}
\setlength{\footskip}{0.3in}
%\fancyhead[L]{\textbf{Devesh Rajpal}}
%\fancyhead[R]{\nouppercase\leftmark}
\usepackage{multicol, titletoc, bookmark, parskip}
%\usepackage{parskip}
\setlength{\parindent}{0.5cm}
\newtheorem{thm}{Theorem}
\newtheorem{defn}{Definition}
\newtheorem{lemma}{Lemma}
\newtheorem*{claim}{Claim}
\newcommand{\R}{\mathbb{R}}
\newcommand{\Z}{\mathbb{Z}}
\newcommand{\Q}{\mathbb{Q}}
\newcommand{\C}{\mathbb{C}}
\newcommand{\half}{\ensuremath{\frac{1}{2}}}
\newcommand{\pt}{\partial_t}
\newcommand{\dt}{\frac{\partial}{\partial t}}


\newcommand{\fg}{\mathfrak{g}}
\title{\LARGE Today I Tried}
\author{\large Devesh Rajpal}
\date{}

\begin{document}

\maketitle
\subsection*{April 2024}

\quad(15/4/24) Today I tried to cast the self similar solutions of $ \alpha $-Gauss flow in the Monge-Ampere type using support function. The equation $ \left< X(x), \nu(x) \right> = cK^{\alpha} $ can be written as 
\[ h = c \left(\frac{\det(\overline{\nabla}^{2}h+\overline{g}h)}{\det(\overline{g})}\right)^{- \alpha} \]
where $ \overline{\nabla}^{2}h $ is the $ 2 $-tensor defined using the standard connection on $ S^{n} $. It is easy to calculate
\[ \overline{\nabla}_{i}\overline{\nabla}_{j}h = \partial_{i}\partial_{j}h- (\overline{\nabla}_{i}\partial_{j})h \]

(16/4/24) Today I tried spherical coordinates on the $ \alpha $-Gauss flow. In the parametrization $ x = \cos \theta \cos \phi, y = \cos \theta \sin \phi, z = \cos \theta $, we have 
\[ \overline{g} = \begin{bmatrix}
    1 & 0 \\
    0 & \cos \theta
\end{bmatrix} \]
 and 
 \[ K^{-1} = \frac{\det \left(\begin{bmatrix}
    h_{\theta \theta} & h_{\theta \phi} - \frac{\tan \theta}{2}h_{\phi} \\
    h_{\theta \phi} - \frac{\tan \theta}{2}h_{\phi} & h_{\phi \phi} - \frac{\cos \theta}{2}h_{\theta}
 \end{bmatrix}
     + h\begin{bmatrix}
        1 & 0 \\
    0 & \cos \theta
     \end{bmatrix}\right)}{\cos \theta}\]
which is quite ugly.

(17/4/24) Today I learned about the homogeneous degree 1 extension of the support function. Let $ h : S^{n} \to \R$ be the support function of a strictly convex hypersurface. We extend this to $ H : \R^{n+1} \to \R $ by defining, 
\[ H(x) = |x| h\left( \frac{x}{|x|}\right).\]
Note that $ H $ is just continuous but not necessarily differentiable at $ 0 $. It is easy to see that $ DH(\lambda x) = DH(x) $. Let $ x \in \R^{n+1} $ and $ v $ be a unit vector, \begin{align*}
    D_{v}H(x) &= h\left( \frac{x}{|x|}\right)D_{v}|x| +|x|D_{v}h\left(\frac{x}{|x|}\right) \\
    & = \frac{v \cdot x}{|x|}h\left( \frac{x}{|x|}\right) + |x|\overline{\nabla}_{v^{T}}h\left( \frac{x}{|x|}\right)
\end{align*}
where $ \overline{\nabla}_{v^{T}} $ is the covariant derivative on $ S^{n} $ in the direction $ v^{T} \in T_{x}S^{n} $. Let $ x \in S^{n} $ and substitute $ v \in \{e_{1}, \dots, e_{n+1}\} $ to get
\[ DH(x) = xh(x)+\overline{\nabla}h(x) \]
which is the inverse of Gauss map! %($ G^{-1}(x) = xh(x)+ \overline{\nabla}h $ as done in the book EGF)!
Thus, $ G^{-1}(x) = DH(x) $, and also weirdly $ D_{x}H(x) = DH(x) $ so the steepest accent is in the normal direction.

(18/4/24) Today I learned about a possible reducible symmetric group to try to construct self-similar solutions of the $ \alpha $-Gauss curvature flow. As considered previously the setup is with support functions. The sphere $ h \equiv 1 $ is an equilibrium point of the normalized $ \alpha $-Gauss flow. The construction of $ \Gamma $ symmetric solutions in Ben's paper is using spherical harmonics (eigenfunctions of the Laplacian) and some general version of the stable/unstable manifold theorem. The linearized version of normalized $ \alpha $-Gauss flow at $ h \equiv 1 $ is given by 
\[ \frac{\partial u}{\partial t} = \alpha( \Delta + n)u+u \]
so if $ \Delta \psi = - \lambda \psi $, then $ h \equiv 1 $ is strictly unstable in the direction $ \psi $ (what does this really mean?). Another important fact is that entropy is a min for unstable direction, the Hessian of entropy at $ h \equiv 1 $ satisfies 
\[ D^{2}Z_{h}(\psi, \psi) > 0. \]

A new idea is to use an affine boost along with the spherical harmonics to possible control the isoperimetric ratio. The considered example of a reducible group was generated by $ x \mapsto -x $ and a $ 3 $-fold rotation symmetric group in $ yz $ plane along with reflection of the triangle (so the dihedral group $ D_{3} $ in the $ yz $ plane). Consider a one-parameter family of affine transformations which stretches the $ x $-direction, 
\[ T_{\lambda} = \begin{bmatrix}
    e^{2\lambda} & 0 & 0 \\
    0 & e^{-\lambda} & 0 \\
    0 & 0 & e^{-\lambda}
\end{bmatrix}. \]
Now we can consider a one parameter family of flows produced by the unstable direction $ \epsilon \psi + T_{\lambda} $ and the expectation is that since the entropy of the solutions with $ \lambda = 0, \infty $ is $ \infty $ (to check) one can possibly use a mountain pass theorem (in homotopies of $ \lambda $ variable) to create a critical point which will be a self-similar solution of the flow.

(19/4/24) Today I learned about the $ \alpha $-Gaussin entropy of a bounded convex hypersurface. In the book it is defined as 

\[ E_\alpha(\mathcal{M}) \doteqdot \begin{cases}\left(\frac{\operatorname{Vol}\left(\mathcal{M}^n\right)}{\left|B^{n+1}\right|}\right)^{\frac{n}{n+1}} \exp \left(\frac{1}{\left|S^n\right|} \int_{\mathcal{M}^n} K \log K d \mu\right) & \text { if } \alpha=1 \\ \left(\frac{\operatorname{Vol}\left(\mathcal{M}^n\right)}{\left|B^{n+1}\right|}\right)^{\frac{n}{n+1}}\left(\frac{1}{\left|S^n\right|} \int_{\mathcal{M}^n} K^\alpha d \mu\right)^{\frac{1}{\alpha-1}} & \text { if } \alpha \neq 1\end{cases} \]

It turns out that the $ \alpha $-Gaussian entropy is non-increasing under $ \alpha $-Gauss flow. The next task is to understand its property on normalized $ \alpha $-Gauss flow.

(22/4/24) Today I learned about the Brunn-Minkowski inequality. It states that for convex bodies $ A, B \subset \R^{n} $ and $ \lambda \in [0,1] $, we have 
\[ \text{Vol}(\lambda A + (1- \lambda)B)^{\frac{1}{n}} \ge \lambda \text{Vol}(A)^{\frac{1}{n}} + (1 - \lambda)\text{Vol}(B)^{\frac{1}{n}} \]
which is same as saying that $ \text{Vol}(\, \cdot \,)^{\frac{1}{n}} $ is a concave function on the set of convex bodies.

(23/4/24) Today I learned about the proof of monotonicity of $ \alpha $-Gaussian entropy under $ \alpha $-Gaussian flow using Brunn-Minkowski inequality.

(24/4/24) Today I tried to finish the monotonicity of entropy proof.

(29/4/24) Today I learned about Jacobi fields in the general context of calculus of variations. Let $ I (u) = \int F(t, u(t),\dot{u}(t)) $ which we want to minimize. 

\subsection*{May 2024}

\quad(3/5/24) Today I learned about Mountain Pass Theorem(MPT). 

(18/5/24) Today I learned about Perron's method of subharmonic function. The Green's function for balls can be obtained with involutions. This allows to solve the Dirichlet problem on balls. Perron's idea is to define a definition of subharmonic function using this result which so that subharmonic functions strong maximum principle (infact the mean value property). 

Let $ \Omega $ be a domain, a function $ u \in C^{0}(\Omega) $ is said to be \textbf{subharmonic} if for every ball $ B \subset \subset \Omega $ and every harmonic function $ h $ on $ B $ with $ u \le h $ on $ \partial B $, we have 
\[ u \le h \text{ on } B. \]
It is easy to see that $ u $ satisfy the mean value property, hence also the strong maximum principle. 
\end{document}